\documentclass[11pt]{article}

    \usepackage[breakable]{tcolorbox}
    \usepackage{parskip} % Stop auto-indenting (to mimic markdown behaviour)
    
    \usepackage{iftex}
    \ifPDFTeX
    	\usepackage[T1]{fontenc}
    	\usepackage{mathpazo}
    \else
    	\usepackage{fontspec}
    \fi

    % Basic figure setup, for now with no caption control since it's done
    % automatically by Pandoc (which extracts ![](path) syntax from Markdown).
    \usepackage{graphicx}
    % Maintain compatibility with old templates. Remove in nbconvert 6.0
    \let\Oldincludegraphics\includegraphics
    % Ensure that by default, figures have no caption (until we provide a
    % proper Figure object with a Caption API and a way to capture that
    % in the conversion process - todo).
    \usepackage{caption}
    \DeclareCaptionFormat{nocaption}{}
    \captionsetup{format=nocaption,aboveskip=0pt,belowskip=0pt}

    \usepackage[Export]{adjustbox} % Used to constrain images to a maximum size
    \adjustboxset{max size={0.9\linewidth}{0.9\paperheight}}
    \usepackage{float}
    \floatplacement{figure}{H} % forces figures to be placed at the correct location
    \usepackage{xcolor} % Allow colors to be defined
    \usepackage{enumerate} % Needed for markdown enumerations to work
    \usepackage{geometry} % Used to adjust the document margins
    \usepackage{amsmath} % Equations
    \usepackage{amssymb} % Equations
    \usepackage{textcomp} % defines textquotesingle
    % Hack from http://tex.stackexchange.com/a/47451/13684:
    \AtBeginDocument{%
        \def\PYZsq{\textquotesingle}% Upright quotes in Pygmentized code
    }
    \usepackage{upquote} % Upright quotes for verbatim code
    \usepackage{eurosym} % defines \euro
    \usepackage[mathletters]{ucs} % Extended unicode (utf-8) support
    \usepackage{fancyvrb} % verbatim replacement that allows latex

    % The hyperref package gives us a pdf with properly built
    % internal navigation ('pdf bookmarks' for the table of contents,
    % internal cross-reference links, web links for URLs, etc.)
    \usepackage{hyperref}
    % The default LaTeX title has an obnoxious amount of whitespace. By default,
    % titling removes some of it. It also provides customization options.
    \usepackage{titling}
    \usepackage{longtable} % longtable support required by pandoc >1.10
    \usepackage{booktabs}  % table support for pandoc > 1.12.2
    \usepackage[inline]{enumitem} % IRkernel/repr support (it uses the enumerate* environment)
    \usepackage[normalem]{ulem} % ulem is needed to support strikethroughs (\sout)
                                % normalem makes italics be italics, not underlines
    \usepackage{mathrsfs}
    

    
    % Colors for the hyperref package
    \definecolor{urlcolor}{rgb}{0,.145,.698}
    \definecolor{linkcolor}{rgb}{.71,0.21,0.01}
    \definecolor{citecolor}{rgb}{.12,.54,.11}

    % ANSI colors
    \definecolor{ansi-black}{HTML}{3E424D}
    \definecolor{ansi-black-intense}{HTML}{282C36}
    \definecolor{ansi-red}{HTML}{E75C58}
    \definecolor{ansi-red-intense}{HTML}{B22B31}
    \definecolor{ansi-green}{HTML}{00A250}
    \definecolor{ansi-green-intense}{HTML}{007427}
    \definecolor{ansi-yellow}{HTML}{DDB62B}
    \definecolor{ansi-yellow-intense}{HTML}{B27D12}
    \definecolor{ansi-blue}{HTML}{208FFB}
    \definecolor{ansi-blue-intense}{HTML}{0065CA}
    \definecolor{ansi-magenta}{HTML}{D160C4}
    \definecolor{ansi-magenta-intense}{HTML}{A03196}
    \definecolor{ansi-cyan}{HTML}{60C6C8}
    \definecolor{ansi-cyan-intense}{HTML}{258F8F}
    \definecolor{ansi-white}{HTML}{C5C1B4}
    \definecolor{ansi-white-intense}{HTML}{A1A6B2}
    \definecolor{ansi-default-inverse-fg}{HTML}{FFFFFF}
    \definecolor{ansi-default-inverse-bg}{HTML}{000000}

    % commands and environments needed by pandoc snippets
    % extracted from the output of `pandoc -s`
    \providecommand{\tightlist}{%
      \setlength{\itemsep}{0pt}\setlength{\parskip}{0pt}}
    \DefineVerbatimEnvironment{Highlighting}{Verbatim}{commandchars=\\\{\}}
    % Add ',fontsize=\small' for more characters per line
    \newenvironment{Shaded}{}{}
    \newcommand{\KeywordTok}[1]{\textcolor[rgb]{0.00,0.44,0.13}{\textbf{{#1}}}}
    \newcommand{\DataTypeTok}[1]{\textcolor[rgb]{0.56,0.13,0.00}{{#1}}}
    \newcommand{\DecValTok}[1]{\textcolor[rgb]{0.25,0.63,0.44}{{#1}}}
    \newcommand{\BaseNTok}[1]{\textcolor[rgb]{0.25,0.63,0.44}{{#1}}}
    \newcommand{\FloatTok}[1]{\textcolor[rgb]{0.25,0.63,0.44}{{#1}}}
    \newcommand{\CharTok}[1]{\textcolor[rgb]{0.25,0.44,0.63}{{#1}}}
    \newcommand{\StringTok}[1]{\textcolor[rgb]{0.25,0.44,0.63}{{#1}}}
    \newcommand{\CommentTok}[1]{\textcolor[rgb]{0.38,0.63,0.69}{\textit{{#1}}}}
    \newcommand{\OtherTok}[1]{\textcolor[rgb]{0.00,0.44,0.13}{{#1}}}
    \newcommand{\AlertTok}[1]{\textcolor[rgb]{1.00,0.00,0.00}{\textbf{{#1}}}}
    \newcommand{\FunctionTok}[1]{\textcolor[rgb]{0.02,0.16,0.49}{{#1}}}
    \newcommand{\RegionMarkerTok}[1]{{#1}}
    \newcommand{\ErrorTok}[1]{\textcolor[rgb]{1.00,0.00,0.00}{\textbf{{#1}}}}
    \newcommand{\NormalTok}[1]{{#1}}
    
    % Additional commands for more recent versions of Pandoc
    \newcommand{\ConstantTok}[1]{\textcolor[rgb]{0.53,0.00,0.00}{{#1}}}
    \newcommand{\SpecialCharTok}[1]{\textcolor[rgb]{0.25,0.44,0.63}{{#1}}}
    \newcommand{\VerbatimStringTok}[1]{\textcolor[rgb]{0.25,0.44,0.63}{{#1}}}
    \newcommand{\SpecialStringTok}[1]{\textcolor[rgb]{0.73,0.40,0.53}{{#1}}}
    \newcommand{\ImportTok}[1]{{#1}}
    \newcommand{\DocumentationTok}[1]{\textcolor[rgb]{0.73,0.13,0.13}{\textit{{#1}}}}
    \newcommand{\AnnotationTok}[1]{\textcolor[rgb]{0.38,0.63,0.69}{\textbf{\textit{{#1}}}}}
    \newcommand{\CommentVarTok}[1]{\textcolor[rgb]{0.38,0.63,0.69}{\textbf{\textit{{#1}}}}}
    \newcommand{\VariableTok}[1]{\textcolor[rgb]{0.10,0.09,0.49}{{#1}}}
    \newcommand{\ControlFlowTok}[1]{\textcolor[rgb]{0.00,0.44,0.13}{\textbf{{#1}}}}
    \newcommand{\OperatorTok}[1]{\textcolor[rgb]{0.40,0.40,0.40}{{#1}}}
    \newcommand{\BuiltInTok}[1]{{#1}}
    \newcommand{\ExtensionTok}[1]{{#1}}
    \newcommand{\PreprocessorTok}[1]{\textcolor[rgb]{0.74,0.48,0.00}{{#1}}}
    \newcommand{\AttributeTok}[1]{\textcolor[rgb]{0.49,0.56,0.16}{{#1}}}
    \newcommand{\InformationTok}[1]{\textcolor[rgb]{0.38,0.63,0.69}{\textbf{\textit{{#1}}}}}
    \newcommand{\WarningTok}[1]{\textcolor[rgb]{0.38,0.63,0.69}{\textbf{\textit{{#1}}}}}
    
    
    % Define a nice break command that doesn't care if a line doesn't already
    % exist.
    \def\br{\hspace*{\fill} \\* }
    % Math Jax compatibility definitions
    \def\gt{>}
    \def\lt{<}
    \let\Oldtex\TeX
    \let\Oldlatex\LaTeX
    \renewcommand{\TeX}{\textrm{\Oldtex}}
    \renewcommand{\LaTeX}{\textrm{\Oldlatex}}
    % Document parameters
    % Document title
    \title{%
    Preprocessing NLP Using NLTK Package \\
    \large CSE 401: Artificial Intelligence }

    \author{%
      Utkarsh Gupta \\
      \small A2305217557 \\
      \small 7CSE 8Y}
    \date{October 04th, 2020}


% Pygments definitions
\makeatletter
\def\PY@reset{\let\PY@it=\relax \let\PY@bf=\relax%
    \let\PY@ul=\relax \let\PY@tc=\relax%
    \let\PY@bc=\relax \let\PY@ff=\relax}
\def\PY@tok#1{\csname PY@tok@#1\endcsname}
\def\PY@toks#1+{\ifx\relax#1\empty\else%
    \PY@tok{#1}\expandafter\PY@toks\fi}
\def\PY@do#1{\PY@bc{\PY@tc{\PY@ul{%
    \PY@it{\PY@bf{\PY@ff{#1}}}}}}}
\def\PY#1#2{\PY@reset\PY@toks#1+\relax+\PY@do{#2}}

\expandafter\def\csname PY@tok@w\endcsname{\def\PY@tc##1{\textcolor[rgb]{0.73,0.73,0.73}{##1}}}
\expandafter\def\csname PY@tok@c\endcsname{\let\PY@it=\textit\def\PY@tc##1{\textcolor[rgb]{0.25,0.50,0.50}{##1}}}
\expandafter\def\csname PY@tok@cp\endcsname{\def\PY@tc##1{\textcolor[rgb]{0.74,0.48,0.00}{##1}}}
\expandafter\def\csname PY@tok@k\endcsname{\let\PY@bf=\textbf\def\PY@tc##1{\textcolor[rgb]{0.00,0.50,0.00}{##1}}}
\expandafter\def\csname PY@tok@kp\endcsname{\def\PY@tc##1{\textcolor[rgb]{0.00,0.50,0.00}{##1}}}
\expandafter\def\csname PY@tok@kt\endcsname{\def\PY@tc##1{\textcolor[rgb]{0.69,0.00,0.25}{##1}}}
\expandafter\def\csname PY@tok@o\endcsname{\def\PY@tc##1{\textcolor[rgb]{0.40,0.40,0.40}{##1}}}
\expandafter\def\csname PY@tok@ow\endcsname{\let\PY@bf=\textbf\def\PY@tc##1{\textcolor[rgb]{0.67,0.13,1.00}{##1}}}
\expandafter\def\csname PY@tok@nb\endcsname{\def\PY@tc##1{\textcolor[rgb]{0.00,0.50,0.00}{##1}}}
\expandafter\def\csname PY@tok@nf\endcsname{\def\PY@tc##1{\textcolor[rgb]{0.00,0.00,1.00}{##1}}}
\expandafter\def\csname PY@tok@nc\endcsname{\let\PY@bf=\textbf\def\PY@tc##1{\textcolor[rgb]{0.00,0.00,1.00}{##1}}}
\expandafter\def\csname PY@tok@nn\endcsname{\let\PY@bf=\textbf\def\PY@tc##1{\textcolor[rgb]{0.00,0.00,1.00}{##1}}}
\expandafter\def\csname PY@tok@ne\endcsname{\let\PY@bf=\textbf\def\PY@tc##1{\textcolor[rgb]{0.82,0.25,0.23}{##1}}}
\expandafter\def\csname PY@tok@nv\endcsname{\def\PY@tc##1{\textcolor[rgb]{0.10,0.09,0.49}{##1}}}
\expandafter\def\csname PY@tok@no\endcsname{\def\PY@tc##1{\textcolor[rgb]{0.53,0.00,0.00}{##1}}}
\expandafter\def\csname PY@tok@nl\endcsname{\def\PY@tc##1{\textcolor[rgb]{0.63,0.63,0.00}{##1}}}
\expandafter\def\csname PY@tok@ni\endcsname{\let\PY@bf=\textbf\def\PY@tc##1{\textcolor[rgb]{0.60,0.60,0.60}{##1}}}
\expandafter\def\csname PY@tok@na\endcsname{\def\PY@tc##1{\textcolor[rgb]{0.49,0.56,0.16}{##1}}}
\expandafter\def\csname PY@tok@nt\endcsname{\let\PY@bf=\textbf\def\PY@tc##1{\textcolor[rgb]{0.00,0.50,0.00}{##1}}}
\expandafter\def\csname PY@tok@nd\endcsname{\def\PY@tc##1{\textcolor[rgb]{0.67,0.13,1.00}{##1}}}
\expandafter\def\csname PY@tok@s\endcsname{\def\PY@tc##1{\textcolor[rgb]{0.73,0.13,0.13}{##1}}}
\expandafter\def\csname PY@tok@sd\endcsname{\let\PY@it=\textit\def\PY@tc##1{\textcolor[rgb]{0.73,0.13,0.13}{##1}}}
\expandafter\def\csname PY@tok@si\endcsname{\let\PY@bf=\textbf\def\PY@tc##1{\textcolor[rgb]{0.73,0.40,0.53}{##1}}}
\expandafter\def\csname PY@tok@se\endcsname{\let\PY@bf=\textbf\def\PY@tc##1{\textcolor[rgb]{0.73,0.40,0.13}{##1}}}
\expandafter\def\csname PY@tok@sr\endcsname{\def\PY@tc##1{\textcolor[rgb]{0.73,0.40,0.53}{##1}}}
\expandafter\def\csname PY@tok@ss\endcsname{\def\PY@tc##1{\textcolor[rgb]{0.10,0.09,0.49}{##1}}}
\expandafter\def\csname PY@tok@sx\endcsname{\def\PY@tc##1{\textcolor[rgb]{0.00,0.50,0.00}{##1}}}
\expandafter\def\csname PY@tok@m\endcsname{\def\PY@tc##1{\textcolor[rgb]{0.40,0.40,0.40}{##1}}}
\expandafter\def\csname PY@tok@gh\endcsname{\let\PY@bf=\textbf\def\PY@tc##1{\textcolor[rgb]{0.00,0.00,0.50}{##1}}}
\expandafter\def\csname PY@tok@gu\endcsname{\let\PY@bf=\textbf\def\PY@tc##1{\textcolor[rgb]{0.50,0.00,0.50}{##1}}}
\expandafter\def\csname PY@tok@gd\endcsname{\def\PY@tc##1{\textcolor[rgb]{0.63,0.00,0.00}{##1}}}
\expandafter\def\csname PY@tok@gi\endcsname{\def\PY@tc##1{\textcolor[rgb]{0.00,0.63,0.00}{##1}}}
\expandafter\def\csname PY@tok@gr\endcsname{\def\PY@tc##1{\textcolor[rgb]{1.00,0.00,0.00}{##1}}}
\expandafter\def\csname PY@tok@ge\endcsname{\let\PY@it=\textit}
\expandafter\def\csname PY@tok@gs\endcsname{\let\PY@bf=\textbf}
\expandafter\def\csname PY@tok@gp\endcsname{\let\PY@bf=\textbf\def\PY@tc##1{\textcolor[rgb]{0.00,0.00,0.50}{##1}}}
\expandafter\def\csname PY@tok@go\endcsname{\def\PY@tc##1{\textcolor[rgb]{0.53,0.53,0.53}{##1}}}
\expandafter\def\csname PY@tok@gt\endcsname{\def\PY@tc##1{\textcolor[rgb]{0.00,0.27,0.87}{##1}}}
\expandafter\def\csname PY@tok@err\endcsname{\def\PY@bc##1{\setlength{\fboxsep}{0pt}\fcolorbox[rgb]{1.00,0.00,0.00}{1,1,1}{\strut ##1}}}
\expandafter\def\csname PY@tok@kc\endcsname{\let\PY@bf=\textbf\def\PY@tc##1{\textcolor[rgb]{0.00,0.50,0.00}{##1}}}
\expandafter\def\csname PY@tok@kd\endcsname{\let\PY@bf=\textbf\def\PY@tc##1{\textcolor[rgb]{0.00,0.50,0.00}{##1}}}
\expandafter\def\csname PY@tok@kn\endcsname{\let\PY@bf=\textbf\def\PY@tc##1{\textcolor[rgb]{0.00,0.50,0.00}{##1}}}
\expandafter\def\csname PY@tok@kr\endcsname{\let\PY@bf=\textbf\def\PY@tc##1{\textcolor[rgb]{0.00,0.50,0.00}{##1}}}
\expandafter\def\csname PY@tok@bp\endcsname{\def\PY@tc##1{\textcolor[rgb]{0.00,0.50,0.00}{##1}}}
\expandafter\def\csname PY@tok@fm\endcsname{\def\PY@tc##1{\textcolor[rgb]{0.00,0.00,1.00}{##1}}}
\expandafter\def\csname PY@tok@vc\endcsname{\def\PY@tc##1{\textcolor[rgb]{0.10,0.09,0.49}{##1}}}
\expandafter\def\csname PY@tok@vg\endcsname{\def\PY@tc##1{\textcolor[rgb]{0.10,0.09,0.49}{##1}}}
\expandafter\def\csname PY@tok@vi\endcsname{\def\PY@tc##1{\textcolor[rgb]{0.10,0.09,0.49}{##1}}}
\expandafter\def\csname PY@tok@vm\endcsname{\def\PY@tc##1{\textcolor[rgb]{0.10,0.09,0.49}{##1}}}
\expandafter\def\csname PY@tok@sa\endcsname{\def\PY@tc##1{\textcolor[rgb]{0.73,0.13,0.13}{##1}}}
\expandafter\def\csname PY@tok@sb\endcsname{\def\PY@tc##1{\textcolor[rgb]{0.73,0.13,0.13}{##1}}}
\expandafter\def\csname PY@tok@sc\endcsname{\def\PY@tc##1{\textcolor[rgb]{0.73,0.13,0.13}{##1}}}
\expandafter\def\csname PY@tok@dl\endcsname{\def\PY@tc##1{\textcolor[rgb]{0.73,0.13,0.13}{##1}}}
\expandafter\def\csname PY@tok@s2\endcsname{\def\PY@tc##1{\textcolor[rgb]{0.73,0.13,0.13}{##1}}}
\expandafter\def\csname PY@tok@sh\endcsname{\def\PY@tc##1{\textcolor[rgb]{0.73,0.13,0.13}{##1}}}
\expandafter\def\csname PY@tok@s1\endcsname{\def\PY@tc##1{\textcolor[rgb]{0.73,0.13,0.13}{##1}}}
\expandafter\def\csname PY@tok@mb\endcsname{\def\PY@tc##1{\textcolor[rgb]{0.40,0.40,0.40}{##1}}}
\expandafter\def\csname PY@tok@mf\endcsname{\def\PY@tc##1{\textcolor[rgb]{0.40,0.40,0.40}{##1}}}
\expandafter\def\csname PY@tok@mh\endcsname{\def\PY@tc##1{\textcolor[rgb]{0.40,0.40,0.40}{##1}}}
\expandafter\def\csname PY@tok@mi\endcsname{\def\PY@tc##1{\textcolor[rgb]{0.40,0.40,0.40}{##1}}}
\expandafter\def\csname PY@tok@il\endcsname{\def\PY@tc##1{\textcolor[rgb]{0.40,0.40,0.40}{##1}}}
\expandafter\def\csname PY@tok@mo\endcsname{\def\PY@tc##1{\textcolor[rgb]{0.40,0.40,0.40}{##1}}}
\expandafter\def\csname PY@tok@ch\endcsname{\let\PY@it=\textit\def\PY@tc##1{\textcolor[rgb]{0.25,0.50,0.50}{##1}}}
\expandafter\def\csname PY@tok@cm\endcsname{\let\PY@it=\textit\def\PY@tc##1{\textcolor[rgb]{0.25,0.50,0.50}{##1}}}
\expandafter\def\csname PY@tok@cpf\endcsname{\let\PY@it=\textit\def\PY@tc##1{\textcolor[rgb]{0.25,0.50,0.50}{##1}}}
\expandafter\def\csname PY@tok@c1\endcsname{\let\PY@it=\textit\def\PY@tc##1{\textcolor[rgb]{0.25,0.50,0.50}{##1}}}
\expandafter\def\csname PY@tok@cs\endcsname{\let\PY@it=\textit\def\PY@tc##1{\textcolor[rgb]{0.25,0.50,0.50}{##1}}}

\def\PYZbs{\char`\\}
\def\PYZus{\char`\_}
\def\PYZob{\char`\{}
\def\PYZcb{\char`\}}
\def\PYZca{\char`\^}
\def\PYZam{\char`\&}
\def\PYZlt{\char`\<}
\def\PYZgt{\char`\>}
\def\PYZsh{\char`\#}
\def\PYZpc{\char`\%}
\def\PYZdl{\char`\$}
\def\PYZhy{\char`\-}
\def\PYZsq{\char`\'}
\def\PYZdq{\char`\"}
\def\PYZti{\char`\~}
% for compatibility with earlier versions
\def\PYZat{@}
\def\PYZlb{[}
\def\PYZrb{]}
\makeatother


    % For linebreaks inside Verbatim environment from package fancyvrb. 
    \makeatletter
        \newbox\Wrappedcontinuationbox 
        \newbox\Wrappedvisiblespacebox 
        \newcommand*\Wrappedvisiblespace {\textcolor{red}{\textvisiblespace}} 
        \newcommand*\Wrappedcontinuationsymbol {\textcolor{red}{\llap{\tiny$\m@th\hookrightarrow$}}} 
        \newcommand*\Wrappedcontinuationindent {3ex } 
        \newcommand*\Wrappedafterbreak {\kern\Wrappedcontinuationindent\copy\Wrappedcontinuationbox} 
        % Take advantage of the already applied Pygments mark-up to insert 
        % potential linebreaks for TeX processing. 
        %        {, <, #, %, $, ' and ": go to next line. 
        %        _, }, ^, &, >, - and ~: stay at end of broken line. 
        % Use of \textquotesingle for straight quote. 
        \newcommand*\Wrappedbreaksatspecials {% 
            \def\PYGZus{\discretionary{\char`\_}{\Wrappedafterbreak}{\char`\_}}% 
            \def\PYGZob{\discretionary{}{\Wrappedafterbreak\char`\{}{\char`\{}}% 
            \def\PYGZcb{\discretionary{\char`\}}{\Wrappedafterbreak}{\char`\}}}% 
            \def\PYGZca{\discretionary{\char`\^}{\Wrappedafterbreak}{\char`\^}}% 
            \def\PYGZam{\discretionary{\char`\&}{\Wrappedafterbreak}{\char`\&}}% 
            \def\PYGZlt{\discretionary{}{\Wrappedafterbreak\char`\<}{\char`\<}}% 
            \def\PYGZgt{\discretionary{\char`\>}{\Wrappedafterbreak}{\char`\>}}% 
            \def\PYGZsh{\discretionary{}{\Wrappedafterbreak\char`\#}{\char`\#}}% 
            \def\PYGZpc{\discretionary{}{\Wrappedafterbreak\char`\%}{\char`\%}}% 
            \def\PYGZdl{\discretionary{}{\Wrappedafterbreak\char`\$}{\char`\$}}% 
            \def\PYGZhy{\discretionary{\char`\-}{\Wrappedafterbreak}{\char`\-}}% 
            \def\PYGZsq{\discretionary{}{\Wrappedafterbreak\textquotesingle}{\textquotesingle}}% 
            \def\PYGZdq{\discretionary{}{\Wrappedafterbreak\char`\"}{\char`\"}}% 
            \def\PYGZti{\discretionary{\char`\~}{\Wrappedafterbreak}{\char`\~}}% 
        } 
        % Some characters . , ; ? ! / are not pygmentized. 
        % This macro makes them "active" and they will insert potential linebreaks 
        \newcommand*\Wrappedbreaksatpunct {% 
            \lccode`\~`\.\lowercase{\def~}{\discretionary{\hbox{\char`\.}}{\Wrappedafterbreak}{\hbox{\char`\.}}}% 
            \lccode`\~`\,\lowercase{\def~}{\discretionary{\hbox{\char`\,}}{\Wrappedafterbreak}{\hbox{\char`\,}}}% 
            \lccode`\~`\;\lowercase{\def~}{\discretionary{\hbox{\char`\;}}{\Wrappedafterbreak}{\hbox{\char`\;}}}% 
            \lccode`\~`\:\lowercase{\def~}{\discretionary{\hbox{\char`\:}}{\Wrappedafterbreak}{\hbox{\char`\:}}}% 
            \lccode`\~`\?\lowercase{\def~}{\discretionary{\hbox{\char`\?}}{\Wrappedafterbreak}{\hbox{\char`\?}}}% 
            \lccode`\~`\!\lowercase{\def~}{\discretionary{\hbox{\char`\!}}{\Wrappedafterbreak}{\hbox{\char`\!}}}% 
            \lccode`\~`\/\lowercase{\def~}{\discretionary{\hbox{\char`\/}}{\Wrappedafterbreak}{\hbox{\char`\/}}}% 
            \catcode`\.\active
            \catcode`\,\active 
            \catcode`\;\active
            \catcode`\:\active
            \catcode`\?\active
            \catcode`\!\active
            \catcode`\/\active 
            \lccode`\~`\~ 	
        }
    \makeatother

    \let\OriginalVerbatim=\Verbatim
    \makeatletter
    \renewcommand{\Verbatim}[1][1]{%
        %\parskip\z@skip
        \sbox\Wrappedcontinuationbox {\Wrappedcontinuationsymbol}%
        \sbox\Wrappedvisiblespacebox {\FV@SetupFont\Wrappedvisiblespace}%
        \def\FancyVerbFormatLine ##1{\hsize\linewidth
            \vtop{\raggedright\hyphenpenalty\z@\exhyphenpenalty\z@
                \doublehyphendemerits\z@\finalhyphendemerits\z@
                \strut ##1\strut}%
        }%
        % If the linebreak is at a space, the latter will be displayed as visible
        % space at end of first line, and a continuation symbol starts next line.
        % Stretch/shrink are however usually zero for typewriter font.
        \def\FV@Space {%
            \nobreak\hskip\z@ plus\fontdimen3\font minus\fontdimen4\font
            \discretionary{\copy\Wrappedvisiblespacebox}{\Wrappedafterbreak}
            {\kern\fontdimen2\font}%
        }%
        
        % Allow breaks at special characters using \PYG... macros.
        \Wrappedbreaksatspecials
        % Breaks at punctuation characters . , ; ? ! and / need catcode=\active 	
        \OriginalVerbatim[#1,codes*=\Wrappedbreaksatpunct]%
    }
    \makeatother

    % Exact colors from NB
    \definecolor{incolor}{HTML}{303F9F}
    \definecolor{outcolor}{HTML}{D84315}
    \definecolor{cellborder}{HTML}{CFCFCF}
    \definecolor{cellbackground}{HTML}{F7F7F7}
    
    % prompt
    \makeatletter
    \newcommand{\boxspacing}{\kern\kvtcb@left@rule\kern\kvtcb@boxsep}
    \makeatother
    \newcommand{\prompt}[4]{
        \ttfamily\llap{{\color{#2}[#3]:\hspace{3pt}#4}}\vspace{-\baselineskip}
    }
    

    
    % Prevent overflowing lines due to hard-to-break entities
    \sloppy 
    % Setup hyperref package
    \hypersetup{
      breaklinks=true,  % so long urls are correctly broken across lines
      colorlinks=true,
      urlcolor=urlcolor,
      linkcolor=linkcolor,
      citecolor=citecolor,
      }
    % Slightly bigger margins than the latex defaults
    
    \geometry{verbose,tmargin=1in,bmargin=1in,lmargin=1in,rmargin=1in}
    
    

\begin{document}
    
    \maketitle
    

    \hypertarget{preprocessing}{%
\section{Preprocessing}\label{preprocessing}}

In this lab, we will be exploring how to preprocess tweets for sentiment
analysis. We will provide a function for preprocessing tweets during
this week's assignment, but it is still good to know what is going on
under the hood. By the end of this lab, you will see how to use the
\href{http://www.nltk.org}{NLTK} package to perform a preprocessing
pipeline for Twitter datasets.

    \hypertarget{setup}{%
\subsection{Setup}\label{setup}}

In this lab, we will be using the
\href{http://www.nltk.org/howto/twitter.html}{Natural Language Toolkit
(NLTK)} package, an open-source Python library for natural language
processing. It has modules for collecting, handling, and processing
Twitter data.

For this exercise, we will use a Twitter dataset that comes with NLTK.
This dataset has been manually annotated and serves to establish
baselines for models quickly. Let us import them now as well as a few
other libraries we will be using.

    \begin{tcolorbox}[breakable, size=fbox, boxrule=1pt, pad at break*=1mm,colback=cellbackground, colframe=cellborder]
\prompt{In}{incolor}{1}{\boxspacing}
\begin{Verbatim}[commandchars=\\\{\}]
\PY{c+c1}{\PYZsh{}Import the necessary libraries}
\PY{k+kn}{import} \PY{n+nn}{nltk}                                \PY{c+c1}{\PYZsh{} Python library for NLP}
\PY{k+kn}{from} \PY{n+nn}{nltk}\PY{n+nn}{.}\PY{n+nn}{corpus} \PY{k}{import} \PY{n}{twitter\PYZus{}samples}    \PY{c+c1}{\PYZsh{} sample Twitter dataset from NLTK}
\PY{k+kn}{import} \PY{n+nn}{matplotlib}\PY{n+nn}{.}\PY{n+nn}{pyplot} \PY{k}{as} \PY{n+nn}{plt}            \PY{c+c1}{\PYZsh{} library for visualization}
\PY{k+kn}{import} \PY{n+nn}{random}                              \PY{c+c1}{\PYZsh{} pseudo\PYZhy{}random number generator}
\PY{k+kn}{import} \PY{n+nn}{numpy} \PY{k}{as} \PY{n+nn}{np} 
\end{Verbatim}
\end{tcolorbox}

    \hypertarget{about-the-twitter-dataset}{%
\subsection{About the Twitter dataset}\label{about-the-twitter-dataset}}

The sample dataset from NLTK is separated into positive and negative
tweets. It contains 5000 positive tweets and 5000 negative tweets
exactly. The exact match between these classes is not a coincidence. The
intention is to have a balanced dataset. That does not reflect the real
distributions of positive and negative classes in live Twitter streams.
It is just because balanced datasets simplify the design of most
computational methods that are required for sentiment analysis. However,
it is better to be aware that this balance of classes is artificial. In
a local computer however, you can download the data by doing:

    \begin{tcolorbox}[breakable, size=fbox, boxrule=1pt, pad at break*=1mm,colback=cellbackground, colframe=cellborder]
\prompt{In}{incolor}{2}{\boxspacing}
\begin{Verbatim}[commandchars=\\\{\}]
\PY{c+c1}{\PYZsh{} downloads sample twitter dataset. execute the line below if running on a local machine.}
\PY{n}{nltk}\PY{o}{.}\PY{n}{download}\PY{p}{(}\PY{l+s+s1}{\PYZsq{}}\PY{l+s+s1}{twitter\PYZus{}samples}\PY{l+s+s1}{\PYZsq{}}\PY{p}{)}
\end{Verbatim}
\end{tcolorbox}

    \begin{Verbatim}[commandchars=\\\{\}]
[nltk\_data] Downloading package twitter\_samples to
[nltk\_data]     /home/utkarsh/nltk\_data{\ldots}
[nltk\_data]   Package twitter\_samples is already up-to-date!
    \end{Verbatim}

            \begin{tcolorbox}[breakable, size=fbox, boxrule=.5pt, pad at break*=1mm, opacityfill=0]
\prompt{Out}{outcolor}{2}{\boxspacing}
\begin{Verbatim}[commandchars=\\\{\}]
True
\end{Verbatim}
\end{tcolorbox}
        
    We can load the text fields of the positive and negative tweets by using
the module's \texttt{strings()} method like this:

    \begin{tcolorbox}[breakable, size=fbox, boxrule=1pt, pad at break*=1mm,colback=cellbackground, colframe=cellborder]
\prompt{In}{incolor}{3}{\boxspacing}
\begin{Verbatim}[commandchars=\\\{\}]
\PY{c+c1}{\PYZsh{} select the set of positive and negative tweets}
\PY{n}{all\PYZus{}positive\PYZus{}tweets} \PY{o}{=} \PY{n}{twitter\PYZus{}samples}\PY{o}{.}\PY{n}{strings}\PY{p}{(}\PY{l+s+s1}{\PYZsq{}}\PY{l+s+s1}{positive\PYZus{}tweets.json}\PY{l+s+s1}{\PYZsq{}}\PY{p}{)}
\PY{n}{all\PYZus{}negative\PYZus{}tweets} \PY{o}{=} \PY{n}{twitter\PYZus{}samples}\PY{o}{.}\PY{n}{strings}\PY{p}{(}\PY{l+s+s1}{\PYZsq{}}\PY{l+s+s1}{negative\PYZus{}tweets.json}\PY{l+s+s1}{\PYZsq{}}\PY{p}{)}
\end{Verbatim}
\end{tcolorbox}

    Next, we'll print a report with the number of positive and negative
tweets. It is also essential to know the data structure of the datasets

    \begin{tcolorbox}[breakable, size=fbox, boxrule=1pt, pad at break*=1mm,colback=cellbackground, colframe=cellborder]
\prompt{In}{incolor}{4}{\boxspacing}
\begin{Verbatim}[commandchars=\\\{\}]
\PY{n+nb}{print}\PY{p}{(}\PY{l+s+s1}{\PYZsq{}}\PY{l+s+s1}{Number of positive tweets: }\PY{l+s+s1}{\PYZsq{}}\PY{p}{,} \PY{n+nb}{len}\PY{p}{(}\PY{n}{all\PYZus{}positive\PYZus{}tweets}\PY{p}{)}\PY{p}{)}
\PY{n+nb}{print}\PY{p}{(}\PY{l+s+s1}{\PYZsq{}}\PY{l+s+s1}{Number of negative tweets: }\PY{l+s+s1}{\PYZsq{}}\PY{p}{,} \PY{n+nb}{len}\PY{p}{(}\PY{n}{all\PYZus{}negative\PYZus{}tweets}\PY{p}{)}\PY{p}{)}

\PY{n+nb}{print}\PY{p}{(}\PY{l+s+s1}{\PYZsq{}}\PY{l+s+se}{\PYZbs{}n}\PY{l+s+s1}{The type of all\PYZus{}positive\PYZus{}tweets is: }\PY{l+s+s1}{\PYZsq{}}\PY{p}{,} \PY{n+nb}{type}\PY{p}{(}\PY{n}{all\PYZus{}positive\PYZus{}tweets}\PY{p}{)}\PY{p}{)}
\PY{n+nb}{print}\PY{p}{(}\PY{l+s+s1}{\PYZsq{}}\PY{l+s+s1}{The type of a tweet entry is: }\PY{l+s+s1}{\PYZsq{}}\PY{p}{,} \PY{n+nb}{type}\PY{p}{(}\PY{n}{all\PYZus{}negative\PYZus{}tweets}\PY{p}{[}\PY{l+m+mi}{0}\PY{p}{]}\PY{p}{)}\PY{p}{)}
\end{Verbatim}
\end{tcolorbox}

    \begin{Verbatim}[commandchars=\\\{\}]
Number of positive tweets:  5000
Number of negative tweets:  5000

The type of all\_positive\_tweets is:  <class 'list'>
The type of a tweet entry is:  <class 'str'>
    \end{Verbatim}

    We can see that the data is stored in a list and as you might expect,
individual tweets are stored as strings.

You can make a more visually appealing report by using Matplotlib's
\href{https://matplotlib.org/tutorials/introductory/pyplot.html}{pyplot}
library. Let us see how to create a
\href{https://matplotlib.org/3.2.1/gallery/pie_and_polar_charts/pie_features.html\#sphx-glr-gallery-pie-and-polar-charts-pie-features-py}{pie
chart} to show the same information as above. This simple snippet will
serve you in future visualizations of this kind of data.

    \begin{tcolorbox}[breakable, size=fbox, boxrule=1pt, pad at break*=1mm,colback=cellbackground, colframe=cellborder]
\prompt{In}{incolor}{5}{\boxspacing}
\begin{Verbatim}[commandchars=\\\{\}]
\PY{c+c1}{\PYZsh{}PLOT the positive and negative tweets in a pie\PYZhy{}chart}
\PY{c+c1}{\PYZsh{} Declare a figure with a custom size}
\PY{n}{fig} \PY{o}{=} \PY{n}{plt}\PY{o}{.}\PY{n}{figure}\PY{p}{(}\PY{n}{figsize}\PY{o}{=}\PY{p}{(}\PY{l+m+mi}{5}\PY{p}{,} \PY{l+m+mi}{5}\PY{p}{)}\PY{p}{)}

\PY{c+c1}{\PYZsh{} labels for the two classes}
\PY{n}{labels} \PY{o}{=} \PY{l+s+s1}{\PYZsq{}}\PY{l+s+s1}{Positives}\PY{l+s+s1}{\PYZsq{}}\PY{p}{,} \PY{l+s+s1}{\PYZsq{}}\PY{l+s+s1}{Negative}\PY{l+s+s1}{\PYZsq{}}

\PY{c+c1}{\PYZsh{} Sizes for each slide}
\PY{n}{sizes} \PY{o}{=} \PY{p}{[}\PY{n+nb}{len}\PY{p}{(}\PY{n}{all\PYZus{}positive\PYZus{}tweets}\PY{p}{)}\PY{p}{,} \PY{n+nb}{len}\PY{p}{(}\PY{n}{all\PYZus{}negative\PYZus{}tweets}\PY{p}{)}\PY{p}{]} 

\PY{c+c1}{\PYZsh{} Declare pie chart, where the slices will be ordered and plotted counter\PYZhy{}clockwise:}
\PY{n}{plt}\PY{o}{.}\PY{n}{pie}\PY{p}{(}\PY{n}{sizes}\PY{p}{,} \PY{n}{labels}\PY{o}{=}\PY{n}{labels}\PY{p}{,} \PY{n}{autopct}\PY{o}{=}\PY{l+s+s1}{\PYZsq{}}\PY{l+s+si}{\PYZpc{}1.1f}\PY{l+s+si}{\PYZpc{}\PYZpc{}}\PY{l+s+s1}{\PYZsq{}}\PY{p}{,}
        \PY{n}{shadow}\PY{o}{=}\PY{k+kc}{True}\PY{p}{,} \PY{n}{startangle}\PY{o}{=}\PY{l+m+mi}{90}\PY{p}{)}

\PY{c+c1}{\PYZsh{} Equal aspect ratio ensures that pie is drawn as a circle.}
\PY{n}{plt}\PY{o}{.}\PY{n}{axis}\PY{p}{(}\PY{l+s+s1}{\PYZsq{}}\PY{l+s+s1}{equal}\PY{l+s+s1}{\PYZsq{}}\PY{p}{)}  

\PY{c+c1}{\PYZsh{} Display the chart}
\PY{n}{plt}\PY{o}{.}\PY{n}{show}\PY{p}{(}\PY{p}{)}
\end{Verbatim}
\end{tcolorbox}

    \begin{center}
    \adjustimage{max size={0.9\linewidth}{0.9\paperheight}}{assets/output_11_0.png}
    \end{center}
    { \hspace*{\fill} \\}
    
    \hypertarget{looking-at-raw-texts}{%
\subsection{Looking at raw texts}\label{looking-at-raw-texts}}

Before anything else, we can print a couple of tweets from the dataset
to see how they look. Understanding the data is responsible for 80\% of
the success or failure in data science projects. We can use this time to
observe aspects we'd like to consider when preprocessing our data.

Below, you will print one random positive and one random negative tweet.
We have added a color mark at the beginning of the string to further
distinguish the two.

    \begin{tcolorbox}[breakable, size=fbox, boxrule=1pt, pad at break*=1mm,colback=cellbackground, colframe=cellborder]
\prompt{In}{incolor}{6}{\boxspacing}
\begin{Verbatim}[commandchars=\\\{\}]
\PY{c+c1}{\PYZsh{} Display a random tweet from positive and negative tweet. The size of positive and negative tweets are 5000 each. }
\PY{c+c1}{\PYZsh{} Generate a random number between 0 and 5000 using random.randint()}
\PY{c+c1}{\PYZsh{} print positive in greeen}
\PY{n+nb}{print}\PY{p}{(}\PY{l+s+s1}{\PYZsq{}}\PY{l+s+se}{\PYZbs{}033}\PY{l+s+s1}{[92m}\PY{l+s+s1}{\PYZsq{}} \PY{o}{+} \PY{n}{all\PYZus{}positive\PYZus{}tweets}\PY{p}{[}\PY{n}{random}\PY{o}{.}\PY{n}{randint}\PY{p}{(}\PY{l+m+mi}{0}\PY{p}{,}\PY{l+m+mi}{5000}\PY{p}{)}\PY{p}{]}\PY{p}{)}

\PY{c+c1}{\PYZsh{} print negative in red}
\PY{n+nb}{print}\PY{p}{(}\PY{l+s+s1}{\PYZsq{}}\PY{l+s+se}{\PYZbs{}033}\PY{l+s+s1}{[91m}\PY{l+s+s1}{\PYZsq{}} \PY{o}{+} \PY{n}{all\PYZus{}negative\PYZus{}tweets}\PY{p}{[}\PY{n}{random}\PY{o}{.}\PY{n}{randint}\PY{p}{(}\PY{l+m+mi}{0}\PY{p}{,}\PY{l+m+mi}{5000}\PY{p}{)}\PY{p}{]}\PY{p}{)}
\end{Verbatim}
\end{tcolorbox}

    \begin{Verbatim}[commandchars=\\\{\}]
\textcolor{ansi-green-intense}{i love airports :-):-))
}\textcolor{ansi-red-intense}{I Blame Rantie For All Of This :(}
    \end{Verbatim}

    One observation you may have is the presence of
\href{https://en.wikipedia.org/wiki/Emoticon}{emoticons} and URLs in
many of the tweets. This info will come in handy in the next steps.

    \hypertarget{preprocess-raw-text-for-sentiment-analysis}{%
\subsection{Preprocess raw text for Sentiment
analysis}\label{preprocess-raw-text-for-sentiment-analysis}}

    Data preprocessing is one of the critical steps in any machine learning
project. It includes cleaning and formatting the data before feeding
into a machine learning algorithm. For NLP, the preprocessing steps are
comprised of the following tasks:

\begin{itemize}
\tightlist
\item
  Tokenizing the string
\item
  Lowercasing
\item
  Removing stop words and punctuation
\item
  Stemming
\end{itemize}

The videos explained each of these steps and why they are important.
Let's see how we can do these to a given tweet. We will choose just one
and see how this is transformed by each preprocessing step.

    \begin{tcolorbox}[breakable, size=fbox, boxrule=1pt, pad at break*=1mm,colback=cellbackground, colframe=cellborder]
\prompt{In}{incolor}{7}{\boxspacing}
\begin{Verbatim}[commandchars=\\\{\}]
\PY{c+c1}{\PYZsh{} Our selected sample. Complex enough to exemplify each step}
\PY{n}{tweet} \PY{o}{=} \PY{n}{all\PYZus{}positive\PYZus{}tweets}\PY{p}{[}\PY{l+m+mi}{2277}\PY{p}{]}
\PY{n+nb}{print}\PY{p}{(}\PY{n}{tweet}\PY{p}{)}
\end{Verbatim}
\end{tcolorbox}

    \begin{Verbatim}[commandchars=\\\{\}]
My beautiful sunflowers on a sunny Friday morning off :) \#sunflowers \#favourites
\#happy \#Friday off… https://t.co/3tfYom0N1i
    \end{Verbatim}

    Let's import a few more libraries for this purpose.

    \begin{tcolorbox}[breakable, size=fbox, boxrule=1pt, pad at break*=1mm,colback=cellbackground, colframe=cellborder]
\prompt{In}{incolor}{8}{\boxspacing}
\begin{Verbatim}[commandchars=\\\{\}]
\PY{c+c1}{\PYZsh{} download the stopwords from NLTK}
\PY{n}{nltk}\PY{o}{.}\PY{n}{download}\PY{p}{(}\PY{l+s+s1}{\PYZsq{}}\PY{l+s+s1}{stopwords}\PY{l+s+s1}{\PYZsq{}}\PY{p}{)}
\end{Verbatim}
\end{tcolorbox}

    \begin{Verbatim}[commandchars=\\\{\}]
[nltk\_data] Downloading package stopwords to
[nltk\_data]     /home/utkarsh/nltk\_data{\ldots}
[nltk\_data]   Package stopwords is already up-to-date!
    \end{Verbatim}

            \begin{tcolorbox}[breakable, size=fbox, boxrule=.5pt, pad at break*=1mm, opacityfill=0]
\prompt{Out}{outcolor}{8}{\boxspacing}
\begin{Verbatim}[commandchars=\\\{\}]
True
\end{Verbatim}
\end{tcolorbox}
        
    \begin{tcolorbox}[breakable, size=fbox, boxrule=1pt, pad at break*=1mm,colback=cellbackground, colframe=cellborder]
\prompt{In}{incolor}{9}{\boxspacing}
\begin{Verbatim}[commandchars=\\\{\}]
\PY{c+c1}{\PYZsh{}Import the necessary libraries}
\PY{k+kn}{import} \PY{n+nn}{re}                                  \PY{c+c1}{\PYZsh{} library for regular expression operations}
\PY{k+kn}{import} \PY{n+nn}{string}                              \PY{c+c1}{\PYZsh{} for string operations}

\PY{k+kn}{from} \PY{n+nn}{nltk}\PY{n+nn}{.}\PY{n+nn}{corpus} \PY{k}{import} \PY{n}{stopwords}          \PY{c+c1}{\PYZsh{} module for stop words that come with NLTK}
\PY{k+kn}{from} \PY{n+nn}{nltk}\PY{n+nn}{.}\PY{n+nn}{stem} \PY{k}{import} \PY{n}{PorterStemmer}        \PY{c+c1}{\PYZsh{} module for stemming}
\PY{k+kn}{from} \PY{n+nn}{nltk}\PY{n+nn}{.}\PY{n+nn}{tokenize} \PY{k}{import} \PY{n}{TweetTokenizer}   \PY{c+c1}{\PYZsh{} module for tokenizing strings}
\end{Verbatim}
\end{tcolorbox}

    \hypertarget{remove-hyperlinks-twitter-marks-and-styles}{%
\subsubsection{Remove hyperlinks, Twitter marks and
styles}\label{remove-hyperlinks-twitter-marks-and-styles}}

Since we have a Twitter dataset, we'd like to remove some substrings
commonly used on the platform like the hashtag, retweet marks, and
hyperlinks. We'll use the
\href{https://docs.python.org/3/library/re.html}{re} library to perform
regular expression operations on our tweet. We'll define our search
pattern and use the \texttt{sub()} method to remove matches by
substituting with an empty character
(i.e.~\texttt{\textquotesingle{}\textquotesingle{}})

    \begin{tcolorbox}[breakable, size=fbox, boxrule=1pt, pad at break*=1mm,colback=cellbackground, colframe=cellborder]
\prompt{In}{incolor}{10}{\boxspacing}
\begin{Verbatim}[commandchars=\\\{\}]
\PY{n+nb}{print}\PY{p}{(}\PY{l+s+s1}{\PYZsq{}}\PY{l+s+se}{\PYZbs{}033}\PY{l+s+s1}{[92m}\PY{l+s+s1}{\PYZsq{}} \PY{o}{+} \PY{n}{tweet}\PY{p}{)}
\PY{n+nb}{print}\PY{p}{(}\PY{l+s+s1}{\PYZsq{}}\PY{l+s+se}{\PYZbs{}033}\PY{l+s+s1}{[94m}\PY{l+s+s1}{\PYZsq{}}\PY{p}{)}

\PY{c+c1}{\PYZsh{} remove old style retweet text \PYZdq{}RT\PYZdq{}}
\PY{n}{tweet2} \PY{o}{=} \PY{n}{re}\PY{o}{.}\PY{n}{sub}\PY{p}{(}\PY{l+s+sa}{r}\PY{l+s+s1}{\PYZsq{}}\PY{l+s+s1}{\PYZca{}RT[}\PY{l+s+s1}{\PYZbs{}}\PY{l+s+s1}{s]+}\PY{l+s+s1}{\PYZsq{}}\PY{p}{,} \PY{l+s+s1}{\PYZsq{}}\PY{l+s+s1}{\PYZsq{}}\PY{p}{,} \PY{n}{tweet}\PY{p}{)}

\PY{c+c1}{\PYZsh{} remove hyperlinks}
\PY{n}{tweet2} \PY{o}{=} \PY{n}{re}\PY{o}{.}\PY{n}{sub}\PY{p}{(}\PY{l+s+sa}{r}\PY{l+s+s1}{\PYZsq{}}\PY{l+s+s1}{https?:}\PY{l+s+s1}{\PYZbs{}}\PY{l+s+s1}{/}\PY{l+s+s1}{\PYZbs{}}\PY{l+s+s1}{/.*[}\PY{l+s+s1}{\PYZbs{}}\PY{l+s+s1}{r}\PY{l+s+s1}{\PYZbs{}}\PY{l+s+s1}{n]*}\PY{l+s+s1}{\PYZsq{}}\PY{p}{,} \PY{l+s+s1}{\PYZsq{}}\PY{l+s+s1}{\PYZsq{}}\PY{p}{,} \PY{n}{tweet2}\PY{p}{)}

\PY{c+c1}{\PYZsh{} remove hashtags}
\PY{c+c1}{\PYZsh{} only removing the hash \PYZsh{} sign from the word}
\PY{n}{tweet2} \PY{o}{=} \PY{n}{re}\PY{o}{.}\PY{n}{sub}\PY{p}{(}\PY{l+s+sa}{r}\PY{l+s+s1}{\PYZsq{}}\PY{l+s+s1}{\PYZsh{}}\PY{l+s+s1}{\PYZsq{}}\PY{p}{,} \PY{l+s+s1}{\PYZsq{}}\PY{l+s+s1}{\PYZsq{}}\PY{p}{,} \PY{n}{tweet2}\PY{p}{)}

\PY{n+nb}{print}\PY{p}{(}\PY{n}{tweet2}\PY{p}{)}
\end{Verbatim}
\end{tcolorbox}

    \begin{Verbatim}[commandchars=\\\{\}]
\textcolor{ansi-green-intense}{My beautiful sunflowers on a sunny Friday morning off :) \#sunflowers
\#favourites \#happy \#Friday off… https://t.co/3tfYom0N1i
}\textcolor{ansi-blue-intense}{
My beautiful sunflowers on a sunny Friday morning off :) sunflowers favourites
happy Friday off…}
    \end{Verbatim}

    \hypertarget{tokenize-the-string}{%
\subsubsection{Tokenize the string}\label{tokenize-the-string}}

To tokenize means to split the strings into individual words without
blanks or tabs. In this same step, we will also convert each word in the
string to lower case. The
\href{https://www.nltk.org/api/nltk.tokenize.html\#module-nltk.tokenize.casual}{tokenize}
module from NLTK allows us to do these easily:

    \begin{tcolorbox}[breakable, size=fbox, boxrule=1pt, pad at break*=1mm,colback=cellbackground, colframe=cellborder]
\prompt{In}{incolor}{11}{\boxspacing}
\begin{Verbatim}[commandchars=\\\{\}]
\PY{n+nb}{print}\PY{p}{(}\PY{p}{)}
\PY{n+nb}{print}\PY{p}{(}\PY{l+s+s1}{\PYZsq{}}\PY{l+s+se}{\PYZbs{}033}\PY{l+s+s1}{[92m}\PY{l+s+s1}{\PYZsq{}} \PY{o}{+} \PY{n}{tweet2}\PY{p}{)}
\PY{n+nb}{print}\PY{p}{(}\PY{l+s+s1}{\PYZsq{}}\PY{l+s+se}{\PYZbs{}033}\PY{l+s+s1}{[94m}\PY{l+s+s1}{\PYZsq{}}\PY{p}{)}

\PY{c+c1}{\PYZsh{} instantiate tokenizer class}
\PY{n}{tokenizer} \PY{o}{=} \PY{n}{TweetTokenizer}\PY{p}{(}\PY{n}{preserve\PYZus{}case}\PY{o}{=}\PY{k+kc}{False}\PY{p}{,} \PY{n}{strip\PYZus{}handles}\PY{o}{=}\PY{k+kc}{True}\PY{p}{,}
                               \PY{n}{reduce\PYZus{}len}\PY{o}{=}\PY{k+kc}{True}\PY{p}{)}

\PY{c+c1}{\PYZsh{} tokenize tweets}
\PY{n}{tweet\PYZus{}tokens} \PY{o}{=} \PY{n}{tokenizer}\PY{o}{.}\PY{n}{tokenize}\PY{p}{(}\PY{n}{tweet2}\PY{p}{)}

\PY{n+nb}{print}\PY{p}{(}\PY{p}{)}
\PY{n+nb}{print}\PY{p}{(}\PY{l+s+s1}{\PYZsq{}}\PY{l+s+s1}{Tokenized string:}\PY{l+s+s1}{\PYZsq{}}\PY{p}{)}
\PY{n+nb}{print}\PY{p}{(}\PY{n}{tweet\PYZus{}tokens}\PY{p}{)}
\end{Verbatim}
\end{tcolorbox}

    \begin{Verbatim}[commandchars=\\\{\}]

\textcolor{ansi-green-intense}{My beautiful sunflowers on a sunny Friday morning off :) sunflowers
favourites happy Friday off…
}\textcolor{ansi-blue-intense}{

Tokenized string:
['my', 'beautiful', 'sunflowers', 'on', 'a', 'sunny', 'friday', 'morning',
'off', ':)', 'sunflowers', 'favourites', 'happy', 'friday', 'off', '…']}
    \end{Verbatim}

    \hypertarget{remove-stop-words-and-punctuations}{%
\subsubsection{Remove stop words and
punctuations}\label{remove-stop-words-and-punctuations}}

The next step is to remove stop words and punctuation. Stop words are
words that don't add significant meaning to the text. You'll see the
list provided by NLTK when you run the cells below.

    \begin{tcolorbox}[breakable, size=fbox, boxrule=1pt, pad at break*=1mm,colback=cellbackground, colframe=cellborder]
\prompt{In}{incolor}{12}{\boxspacing}
\begin{Verbatim}[commandchars=\\\{\}]
\PY{c+c1}{\PYZsh{}Import the english stop words list from NLTK}
\PY{n}{stopwords\PYZus{}english} \PY{o}{=} \PY{n}{stopwords}\PY{o}{.}\PY{n}{words}\PY{p}{(}\PY{l+s+s1}{\PYZsq{}}\PY{l+s+s1}{english}\PY{l+s+s1}{\PYZsq{}}\PY{p}{)} 

\PY{n+nb}{print}\PY{p}{(}\PY{l+s+s1}{\PYZsq{}}\PY{l+s+s1}{Stop words}\PY{l+s+se}{\PYZbs{}n}\PY{l+s+s1}{\PYZsq{}}\PY{p}{)}
\PY{n+nb}{print}\PY{p}{(}\PY{n}{stopwords\PYZus{}english}\PY{p}{)}

\PY{n+nb}{print}\PY{p}{(}\PY{l+s+s1}{\PYZsq{}}\PY{l+s+se}{\PYZbs{}n}\PY{l+s+s1}{Punctuation}\PY{l+s+se}{\PYZbs{}n}\PY{l+s+s1}{\PYZsq{}}\PY{p}{)}
\PY{n+nb}{print}\PY{p}{(}\PY{n}{string}\PY{o}{.}\PY{n}{punctuation}\PY{p}{)}
\end{Verbatim}
\end{tcolorbox}

    \begin{Verbatim}[commandchars=\\\{\}]
Stop words

['i', 'me', 'my', 'myself', 'we', 'our', 'ours', 'ourselves', 'you', "you're",
"you've", "you'll", "you'd", 'your', 'yours', 'yourself', 'yourselves', 'he',
'him', 'his', 'himself', 'she', "she's", 'her', 'hers', 'herself', 'it', "it's",
'its', 'itself', 'they', 'them', 'their', 'theirs', 'themselves', 'what',
'which', 'who', 'whom', 'this', 'that', "that'll", 'these', 'those', 'am', 'is',
'are', 'was', 'were', 'be', 'been', 'being', 'have', 'has', 'had', 'having',
'do', 'does', 'did', 'doing', 'a', 'an', 'the', 'and', 'but', 'if', 'or',
'because', 'as', 'until', 'while', 'of', 'at', 'by', 'for', 'with', 'about',
'against', 'between', 'into', 'through', 'during', 'before', 'after', 'above',
'below', 'to', 'from', 'up', 'down', 'in', 'out', 'on', 'off', 'over', 'under',
'again', 'further', 'then', 'once', 'here', 'there', 'when', 'where', 'why',
'how', 'all', 'any', 'both', 'each', 'few', 'more', 'most', 'other', 'some',
'such', 'no', 'nor', 'not', 'only', 'own', 'same', 'so', 'than', 'too', 'very',
's', 't', 'can', 'will', 'just', 'don', "don't", 'should', "should've", 'now',
'd', 'll', 'm', 'o', 're', 've', 'y', 'ain', 'aren', "aren't", 'couldn',
"couldn't", 'didn', "didn't", 'doesn', "doesn't", 'hadn', "hadn't", 'hasn',
"hasn't", 'haven', "haven't", 'isn', "isn't", 'ma', 'mightn', "mightn't",
'mustn', "mustn't", 'needn', "needn't", 'shan', "shan't", 'shouldn',
"shouldn't", 'wasn', "wasn't", 'weren', "weren't", 'won', "won't", 'wouldn',
"wouldn't"]

Punctuation

!"\#\$\%\&'()*+,-./:;<=>?@[\textbackslash{}]\^{}\_`\{|\}\textasciitilde{}
    \end{Verbatim}

    We can see that the stop words list above contains some words that could
be important in some contexts. These could be words like \emph{i, not,
between, because, won, against}. You might need to customize the stop
words list for some applications. For our exercise, we will use the
entire list.

For the punctuation, we saw earlier that certain groupings like `:)' and
`\ldots{}' should be retained when dealing with tweets because they are
used to express emotions. In other contexts, like medical analysis,
these should also be removed.

Time to clean up our tokenized tweet!

    \begin{tcolorbox}[breakable, size=fbox, boxrule=1pt, pad at break*=1mm,colback=cellbackground, colframe=cellborder]
\prompt{In}{incolor}{13}{\boxspacing}
\begin{Verbatim}[commandchars=\\\{\}]
\PY{n+nb}{print}\PY{p}{(}\PY{p}{)}
\PY{n+nb}{print}\PY{p}{(}\PY{l+s+s1}{\PYZsq{}}\PY{l+s+se}{\PYZbs{}033}\PY{l+s+s1}{[92m}\PY{l+s+s1}{\PYZsq{}}\PY{p}{)}
\PY{n+nb}{print}\PY{p}{(}\PY{n}{tweet\PYZus{}tokens}\PY{p}{)}
\PY{n+nb}{print}\PY{p}{(}\PY{l+s+s1}{\PYZsq{}}\PY{l+s+se}{\PYZbs{}033}\PY{l+s+s1}{[94m}\PY{l+s+s1}{\PYZsq{}}\PY{p}{)}

\PY{n}{tweets\PYZus{}clean} \PY{o}{=} \PY{p}{[}\PY{p}{]}

\PY{k}{for} \PY{n}{word} \PY{o+ow}{in} \PY{n}{tweet\PYZus{}tokens}\PY{p}{:} \PY{c+c1}{\PYZsh{} Go through every word in your tokens list}
    \PY{k}{if} \PY{p}{(}\PY{n}{word} \PY{o+ow}{not} \PY{o+ow}{in} \PY{n}{stopwords\PYZus{}english} \PY{o+ow}{and}  \PY{c+c1}{\PYZsh{} remove stopwords}
        \PY{n}{word} \PY{o+ow}{not} \PY{o+ow}{in} \PY{n}{string}\PY{o}{.}\PY{n}{punctuation}\PY{p}{)}\PY{p}{:}  \PY{c+c1}{\PYZsh{} remove punctuation}
        \PY{n}{tweets\PYZus{}clean}\PY{o}{.}\PY{n}{append}\PY{p}{(}\PY{n}{word}\PY{p}{)}

\PY{n+nb}{print}\PY{p}{(}\PY{l+s+s1}{\PYZsq{}}\PY{l+s+s1}{removed stop words and punctuation:}\PY{l+s+s1}{\PYZsq{}}\PY{p}{)}
\PY{n+nb}{print}\PY{p}{(}\PY{n}{tweets\PYZus{}clean}\PY{p}{)}
\end{Verbatim}
\end{tcolorbox}

    \begin{Verbatim}[commandchars=\\\{\}]

\textcolor{ansi-green-intense}{
['my', 'beautiful', 'sunflowers', 'on', 'a', 'sunny', 'friday', 'morning',
'off', ':)', 'sunflowers', 'favourites', 'happy', 'friday', 'off', '…']
}\textcolor{ansi-blue-intense}{
removed stop words and punctuation:
['beautiful', 'sunflowers', 'sunny', 'friday', 'morning', ':)', 'sunflowers',
'favourites', 'happy', 'friday', '…']}
    \end{Verbatim}

    Please note that the words \textbf{happy} and \textbf{sunny} in this
list are correctly spelled.

    \hypertarget{stemming}{%
\subsubsection{Stemming}\label{stemming}}

Stemming is the process of converting a word to its most general form,
or stem. This helps in reducing the size of our vocabulary.

Consider the words: * \textbf{learn} * \textbf{learn}ing *
\textbf{learn}ed * \textbf{learn}t

All these words are stemmed from its common root \textbf{learn}.
However, in some cases, the stemming process produces words that are not
correct spellings of the root word. For example, \textbf{happi} and
\textbf{sunni}. That's because it chooses the most common stem for
related words. For example, we can look at the set of words that
comprises the different forms of happy:

\begin{itemize}
\tightlist
\item
  \textbf{happ}y
\item
  \textbf{happi}ness
\item
  \textbf{happi}er
\end{itemize}

We can see that the prefix \textbf{happi} is more commonly used. We
cannot choose \textbf{happ} because it is the stem of unrelated words
like \textbf{happen}.

NLTK has different modules for stemming and we will be using the
\href{https://www.nltk.org/api/nltk.stem.html\#module-nltk.stem.porter}{PorterStemmer}
module which uses the
\href{https://tartarus.org/martin/PorterStemmer/}{Porter Stemming
Algorithm}. Let's see how we can use it in the cell below.

    \begin{tcolorbox}[breakable, size=fbox, boxrule=1pt, pad at break*=1mm,colback=cellbackground, colframe=cellborder]
\prompt{In}{incolor}{14}{\boxspacing}
\begin{Verbatim}[commandchars=\\\{\}]
\PY{n+nb}{print}\PY{p}{(}\PY{p}{)}
\PY{n+nb}{print}\PY{p}{(}\PY{l+s+s1}{\PYZsq{}}\PY{l+s+se}{\PYZbs{}033}\PY{l+s+s1}{[92m}\PY{l+s+s1}{\PYZsq{}}\PY{p}{)}
\PY{n+nb}{print}\PY{p}{(}\PY{n}{tweets\PYZus{}clean}\PY{p}{)}
\PY{n+nb}{print}\PY{p}{(}\PY{l+s+s1}{\PYZsq{}}\PY{l+s+se}{\PYZbs{}033}\PY{l+s+s1}{[94m}\PY{l+s+s1}{\PYZsq{}}\PY{p}{)}

\PY{c+c1}{\PYZsh{} Instantiate stemming class}
\PY{n}{stemmer} \PY{o}{=} \PY{n}{PorterStemmer}\PY{p}{(}\PY{p}{)} 

\PY{n}{tweets\PYZus{}stem} \PY{o}{=} \PY{p}{[}\PY{p}{]}

\PY{k}{for} \PY{n}{word} \PY{o+ow}{in} \PY{n}{tweets\PYZus{}clean}\PY{p}{:}
    \PY{n}{stem\PYZus{}word} \PY{o}{=} \PY{n}{stemmer}\PY{o}{.}\PY{n}{stem}\PY{p}{(}\PY{n}{word}\PY{p}{)}  \PY{c+c1}{\PYZsh{} stemming word}
    \PY{n}{tweets\PYZus{}stem}\PY{o}{.}\PY{n}{append}\PY{p}{(}\PY{n}{stem\PYZus{}word}\PY{p}{)}  \PY{c+c1}{\PYZsh{} append to the list}

\PY{n+nb}{print}\PY{p}{(}\PY{l+s+s1}{\PYZsq{}}\PY{l+s+s1}{stemmed words:}\PY{l+s+s1}{\PYZsq{}}\PY{p}{)}
\PY{n+nb}{print}\PY{p}{(}\PY{n}{tweets\PYZus{}stem}\PY{p}{)}
\end{Verbatim}
\end{tcolorbox}

    \begin{Verbatim}[commandchars=\\\{\}]

\textcolor{ansi-green-intense}{
['beautiful', 'sunflowers', 'sunny', 'friday', 'morning', ':)', 'sunflowers',
'favourites', 'happy', 'friday', '…']
}\textcolor{ansi-blue-intense}{
stemmed words:
['beauti', 'sunflow', 'sunni', 'friday', 'morn', ':)', 'sunflow', 'favourit',
'happi', 'friday', '…']}
    \end{Verbatim}

    That's it! Now we have a set of words we can feed into to the next stage
of our machine learning project.

    \hypertarget{process_tweet}{%
\subsection{process\_tweet()}\label{process_tweet}}

As shown above, preprocessing consists of multiple steps before you
arrive at the final list of words. We will not ask you to replicate
these however. You will use the function \texttt{process\_tweet(tweet)}
available below.

To obtain the same result as in the previous code cells, you will only
need to call the function \texttt{process\_tweet()}. Let's do that in
the next cell.

    \begin{tcolorbox}[breakable, size=fbox, boxrule=1pt, pad at break*=1mm,colback=cellbackground, colframe=cellborder]
\prompt{In}{incolor}{15}{\boxspacing}
\begin{Verbatim}[commandchars=\\\{\}]
\PY{k}{def} \PY{n+nf}{process\PYZus{}tweet}\PY{p}{(}\PY{n}{tweet}\PY{p}{)}\PY{p}{:}
    \PY{l+s+sd}{\PYZdq{}\PYZdq{}\PYZdq{}Process tweet function.}
\PY{l+s+sd}{    Input:}
\PY{l+s+sd}{        tweet: a string containing a tweet}
\PY{l+s+sd}{    Output:}
\PY{l+s+sd}{        tweets\PYZus{}clean: a list of words containing the processed tweet}

\PY{l+s+sd}{    \PYZdq{}\PYZdq{}\PYZdq{}}
    \PY{n}{stemmer} \PY{o}{=} \PY{n}{PorterStemmer}\PY{p}{(}\PY{p}{)}
    \PY{n}{stopwords\PYZus{}english} \PY{o}{=} \PY{n}{stopwords}\PY{o}{.}\PY{n}{words}\PY{p}{(}\PY{l+s+s1}{\PYZsq{}}\PY{l+s+s1}{english}\PY{l+s+s1}{\PYZsq{}}\PY{p}{)}
    \PY{c+c1}{\PYZsh{} remove stock market tickers like \PYZdl{}GE}
    \PY{n}{tweet} \PY{o}{=} \PY{n}{re}\PY{o}{.}\PY{n}{sub}\PY{p}{(}\PY{l+s+sa}{r}\PY{l+s+s1}{\PYZsq{}}\PY{l+s+s1}{\PYZbs{}}\PY{l+s+s1}{\PYZdl{}}\PY{l+s+s1}{\PYZbs{}}\PY{l+s+s1}{w*}\PY{l+s+s1}{\PYZsq{}}\PY{p}{,} \PY{l+s+s1}{\PYZsq{}}\PY{l+s+s1}{\PYZsq{}}\PY{p}{,} \PY{n}{tweet}\PY{p}{)}
    \PY{c+c1}{\PYZsh{} remove old style retweet text \PYZdq{}RT\PYZdq{}}
    \PY{n}{tweet} \PY{o}{=} \PY{n}{re}\PY{o}{.}\PY{n}{sub}\PY{p}{(}\PY{l+s+sa}{r}\PY{l+s+s1}{\PYZsq{}}\PY{l+s+s1}{\PYZca{}RT[}\PY{l+s+s1}{\PYZbs{}}\PY{l+s+s1}{s]+}\PY{l+s+s1}{\PYZsq{}}\PY{p}{,} \PY{l+s+s1}{\PYZsq{}}\PY{l+s+s1}{\PYZsq{}}\PY{p}{,} \PY{n}{tweet}\PY{p}{)}
    \PY{c+c1}{\PYZsh{} remove hyperlinks}
    \PY{n}{tweet} \PY{o}{=} \PY{n}{re}\PY{o}{.}\PY{n}{sub}\PY{p}{(}\PY{l+s+sa}{r}\PY{l+s+s1}{\PYZsq{}}\PY{l+s+s1}{https?:}\PY{l+s+s1}{\PYZbs{}}\PY{l+s+s1}{/}\PY{l+s+s1}{\PYZbs{}}\PY{l+s+s1}{/.*[}\PY{l+s+s1}{\PYZbs{}}\PY{l+s+s1}{r}\PY{l+s+s1}{\PYZbs{}}\PY{l+s+s1}{n]*}\PY{l+s+s1}{\PYZsq{}}\PY{p}{,} \PY{l+s+s1}{\PYZsq{}}\PY{l+s+s1}{\PYZsq{}}\PY{p}{,} \PY{n}{tweet}\PY{p}{)}
    \PY{c+c1}{\PYZsh{} remove hashtags}
    \PY{c+c1}{\PYZsh{} only removing the hash \PYZsh{} sign from the word}
    \PY{n}{tweet} \PY{o}{=} \PY{n}{re}\PY{o}{.}\PY{n}{sub}\PY{p}{(}\PY{l+s+sa}{r}\PY{l+s+s1}{\PYZsq{}}\PY{l+s+s1}{\PYZsh{}}\PY{l+s+s1}{\PYZsq{}}\PY{p}{,} \PY{l+s+s1}{\PYZsq{}}\PY{l+s+s1}{\PYZsq{}}\PY{p}{,} \PY{n}{tweet}\PY{p}{)}
    \PY{c+c1}{\PYZsh{} tokenize tweets}
    \PY{n}{tokenizer} \PY{o}{=} \PY{n}{TweetTokenizer}\PY{p}{(}\PY{n}{preserve\PYZus{}case}\PY{o}{=}\PY{k+kc}{False}\PY{p}{,} \PY{n}{strip\PYZus{}handles}\PY{o}{=}\PY{k+kc}{True}\PY{p}{,}
                               \PY{n}{reduce\PYZus{}len}\PY{o}{=}\PY{k+kc}{True}\PY{p}{)}
    \PY{n}{tweet\PYZus{}tokens} \PY{o}{=} \PY{n}{tokenizer}\PY{o}{.}\PY{n}{tokenize}\PY{p}{(}\PY{n}{tweet}\PY{p}{)}

    \PY{n}{tweets\PYZus{}clean} \PY{o}{=} \PY{p}{[}\PY{p}{]}
    \PY{k}{for} \PY{n}{word} \PY{o+ow}{in} \PY{n}{tweet\PYZus{}tokens}\PY{p}{:}
        \PY{k}{if} \PY{p}{(}\PY{n}{word} \PY{o+ow}{not} \PY{o+ow}{in} \PY{n}{stopwords\PYZus{}english} \PY{o+ow}{and}  \PY{c+c1}{\PYZsh{} remove stopwords}
                \PY{n}{word} \PY{o+ow}{not} \PY{o+ow}{in} \PY{n}{string}\PY{o}{.}\PY{n}{punctuation}\PY{p}{)}\PY{p}{:}  \PY{c+c1}{\PYZsh{} remove punctuation}
            \PY{c+c1}{\PYZsh{} tweets\PYZus{}clean.append(word)}
            \PY{n}{stem\PYZus{}word} \PY{o}{=} \PY{n}{stemmer}\PY{o}{.}\PY{n}{stem}\PY{p}{(}\PY{n}{word}\PY{p}{)}  \PY{c+c1}{\PYZsh{} stemming word}
            \PY{n}{tweets\PYZus{}clean}\PY{o}{.}\PY{n}{append}\PY{p}{(}\PY{n}{stem\PYZus{}word}\PY{p}{)}

    \PY{k}{return} \PY{n}{tweets\PYZus{}clean}
\end{Verbatim}
\end{tcolorbox}

    \begin{tcolorbox}[breakable, size=fbox, boxrule=1pt, pad at break*=1mm,colback=cellbackground, colframe=cellborder]
\prompt{In}{incolor}{16}{\boxspacing}
\begin{Verbatim}[commandchars=\\\{\}]
\PY{c+c1}{\PYZsh{} choose the same tweet}
\PY{n}{tweet} \PY{o}{=} \PY{n}{all\PYZus{}positive\PYZus{}tweets}\PY{p}{[}\PY{l+m+mi}{2277}\PY{p}{]}

\PY{n+nb}{print}\PY{p}{(}\PY{p}{)}
\PY{n+nb}{print}\PY{p}{(}\PY{l+s+s1}{\PYZsq{}}\PY{l+s+se}{\PYZbs{}033}\PY{l+s+s1}{[92m}\PY{l+s+s1}{\PYZsq{}}\PY{p}{)}
\PY{n+nb}{print}\PY{p}{(}\PY{n}{tweet}\PY{p}{)}
\PY{n+nb}{print}\PY{p}{(}\PY{l+s+s1}{\PYZsq{}}\PY{l+s+se}{\PYZbs{}033}\PY{l+s+s1}{[94m}\PY{l+s+s1}{\PYZsq{}}\PY{p}{)}

\PY{c+c1}{\PYZsh{} call the process\PYZus{}tweet function}
\PY{n}{tweets\PYZus{}stem} \PY{o}{=} \PY{n}{process\PYZus{}tweet}\PY{p}{(}\PY{n}{tweet}\PY{p}{)}

\PY{n+nb}{print}\PY{p}{(}\PY{l+s+s1}{\PYZsq{}}\PY{l+s+s1}{preprocessed tweet:}\PY{l+s+s1}{\PYZsq{}}\PY{p}{)}
\PY{n+nb}{print}\PY{p}{(}\PY{n}{tweets\PYZus{}stem}\PY{p}{)} \PY{c+c1}{\PYZsh{} Print the result}
\end{Verbatim}
\end{tcolorbox}

    \begin{Verbatim}[commandchars=\\\{\}]

\textcolor{ansi-green-intense}{
My beautiful sunflowers on a sunny Friday morning off :) \#sunflowers \#favourites
\#happy \#Friday off… https://t.co/3tfYom0N1i
}\textcolor{ansi-blue-intense}{
preprocessed tweet:
['beauti', 'sunflow', 'sunni', 'friday', 'morn', ':)', 'sunflow', 'favourit',
'happi', 'friday', '…']}
    \end{Verbatim}

    ~\\
    \noindent\rule{16.5cm}{0.4pt}
    ~\newpage

    \hypertarget{building-and-visualizing-word-frequencies}{%
\section{Building and Visualizing word
frequencies}\label{building-and-visualizing-word-frequencies}}

In this lab, we will focus on the \texttt{build\_freqs()} helper
function and visualizing a dataset fed into it. In our goal of tweet
sentiment analysis, this function will build a dictionary where we can
lookup how many times a word appears in the lists of positive or
negative tweets. This will be very helpful when extracting the features
of the dataset in the week's programming assignment. Let's see how this
function is implemented under the hood in this notebook.

    \begin{tcolorbox}[breakable, size=fbox, boxrule=1pt, pad at break*=1mm,colback=cellbackground, colframe=cellborder]
\prompt{In}{incolor}{17}{\boxspacing}
\begin{Verbatim}[commandchars=\\\{\}]
\PY{c+c1}{\PYZsh{} Concatenate the lists, 1st part is the positive tweets followed by the negative}
\PY{n}{tweets} \PY{o}{=} \PY{n}{all\PYZus{}positive\PYZus{}tweets} \PY{o}{+} \PY{n}{all\PYZus{}negative\PYZus{}tweets}

\PY{c+c1}{\PYZsh{} let\PYZsq{}s see how many tweets we have}
\PY{n+nb}{print}\PY{p}{(}\PY{l+s+s2}{\PYZdq{}}\PY{l+s+s2}{Number of tweets: }\PY{l+s+s2}{\PYZdq{}}\PY{p}{,} \PY{n+nb}{len}\PY{p}{(}\PY{n}{tweets}\PY{p}{)}\PY{p}{)}
\end{Verbatim}
\end{tcolorbox}

    \begin{Verbatim}[commandchars=\\\{\}]
Number of tweets:  10000
    \end{Verbatim}

    Next, we will build a labels array that matches the sentiments of our
tweets. This data type works pretty much like a regular list but is
optimized for computations and manipulation. The \texttt{labels} array
will be composed of 10000 elements. The first 5000 will be filled with
\texttt{1} labels denoting positive sentiments, and the next 5000 will
be \texttt{0} labels denoting the opposite. We can do this easily with a
series of operations provided by the \texttt{numpy} library:

\begin{itemize}
\tightlist
\item
  \texttt{np.ones()} - create an array of 1's
\item
  \texttt{np.zeros()} - create an array of 0's
\item
  \texttt{np.append()} - concatenate arrays
\end{itemize}

    \begin{tcolorbox}[breakable, size=fbox, boxrule=1pt, pad at break*=1mm,colback=cellbackground, colframe=cellborder]
\prompt{In}{incolor}{18}{\boxspacing}
\begin{Verbatim}[commandchars=\\\{\}]
\PY{c+c1}{\PYZsh{} make a numpy array representing labels of the tweets}
\PY{n}{labels} \PY{o}{=} \PY{n}{np}\PY{o}{.}\PY{n}{append}\PY{p}{(}\PY{n}{np}\PY{o}{.}\PY{n}{ones}\PY{p}{(}\PY{p}{(}\PY{n+nb}{len}\PY{p}{(}\PY{n}{all\PYZus{}positive\PYZus{}tweets}\PY{p}{)}\PY{p}{)}\PY{p}{)}\PY{p}{,} \PY{n}{np}\PY{o}{.}\PY{n}{zeros}\PY{p}{(}\PY{p}{(}\PY{n+nb}{len}\PY{p}{(}\PY{n}{all\PYZus{}negative\PYZus{}tweets}\PY{p}{)}\PY{p}{)}\PY{p}{)}\PY{p}{)}
\end{Verbatim}
\end{tcolorbox}

    \hypertarget{dictionaries}{%
\subsection{Dictionaries}\label{dictionaries}}

In Python, a dictionary is a mutable and indexed collection. It stores
items as key-value pairs and uses
\href{https://en.wikipedia.org/wiki/Hash_table}{hash tables} underneath
to allow practically constant time lookups. In NLP, dictionaries are
essential because it enables fast retrieval of items or containment
checks even with thousands of entries in the collection.

    \hypertarget{definition}{%
\subsubsection{Definition}\label{definition}}

A dictionary in Python is declared using curly brackets. Look at the
next example:

    \begin{tcolorbox}[breakable, size=fbox, boxrule=1pt, pad at break*=1mm,colback=cellbackground, colframe=cellborder]
\prompt{In}{incolor}{19}{\boxspacing}
\begin{Verbatim}[commandchars=\\\{\}]
\PY{n}{dictionary} \PY{o}{=} \PY{p}{\PYZob{}}\PY{l+s+s1}{\PYZsq{}}\PY{l+s+s1}{key1}\PY{l+s+s1}{\PYZsq{}}\PY{p}{:} \PY{l+m+mi}{1}\PY{p}{,} \PY{l+s+s1}{\PYZsq{}}\PY{l+s+s1}{key2}\PY{l+s+s1}{\PYZsq{}}\PY{p}{:} \PY{l+m+mi}{2}\PY{p}{\PYZcb{}}
\end{Verbatim}
\end{tcolorbox}

    The former line defines a dictionary with two entries. Keys and values
can be almost any type
(\href{https://docs.python.org/3/tutorial/datastructures.html\#dictionaries}{with
a few restriction on keys}), and in this case, we used strings. We can
also use floats, integers, tuples, etc.

\hypertarget{adding-or-editing-entries}{%
\subsubsection{Adding or editing
entries}\label{adding-or-editing-entries}}

New entries can be inserted into dictionaries using square brackets. If
the dictionary already contains the specified key, its value is
overwritten.

    \begin{tcolorbox}[breakable, size=fbox, boxrule=1pt, pad at break*=1mm,colback=cellbackground, colframe=cellborder]
\prompt{In}{incolor}{20}{\boxspacing}
\begin{Verbatim}[commandchars=\\\{\}]
\PY{c+c1}{\PYZsh{} Add a new entry}
\PY{n}{dictionary}\PY{p}{[}\PY{l+s+s1}{\PYZsq{}}\PY{l+s+s1}{key3}\PY{l+s+s1}{\PYZsq{}}\PY{p}{]} \PY{o}{=} \PY{o}{\PYZhy{}}\PY{l+m+mi}{5}

\PY{c+c1}{\PYZsh{} Overwrite the value of key1}
\PY{n}{dictionary}\PY{p}{[}\PY{l+s+s1}{\PYZsq{}}\PY{l+s+s1}{key1}\PY{l+s+s1}{\PYZsq{}}\PY{p}{]} \PY{o}{=} \PY{l+m+mi}{0}

\PY{n+nb}{print}\PY{p}{(}\PY{n}{dictionary}\PY{p}{)}
\end{Verbatim}
\end{tcolorbox}

    \begin{Verbatim}[commandchars=\\\{\}]
\{'key1': 0, 'key2': 2, 'key3': -5\}
    \end{Verbatim}

    \hypertarget{accessing-values-and-lookup-keys}{%
\subsubsection{Accessing values and lookup
keys}\label{accessing-values-and-lookup-keys}}

Performing dictionary lookups and retrieval are common tasks in NLP.
There are two ways to do this:

\begin{itemize}
\tightlist
\item
  Using square bracket notation: This form is allowed if the lookup key
  is in the dictionary. It produces an error otherwise.
\item
  Using the
  \href{https://docs.python.org/3/library/stdtypes.html\#dict.get}{get()}
  method: This allows us to set a default value if the dictionary key
  does not exist.
\end{itemize}

Let us see these in action:

    \begin{tcolorbox}[breakable, size=fbox, boxrule=1pt, pad at break*=1mm,colback=cellbackground, colframe=cellborder]
\prompt{In}{incolor}{21}{\boxspacing}
\begin{Verbatim}[commandchars=\\\{\}]
\PY{c+c1}{\PYZsh{} Square bracket lookup when the key exist}
\PY{n+nb}{print}\PY{p}{(}\PY{n}{dictionary}\PY{p}{[}\PY{l+s+s1}{\PYZsq{}}\PY{l+s+s1}{key2}\PY{l+s+s1}{\PYZsq{}}\PY{p}{]}\PY{p}{)}
\end{Verbatim}
\end{tcolorbox}

    \begin{Verbatim}[commandchars=\\\{\}]
2
    \end{Verbatim}

    However, if the key is missing, the operation produce an erro

    \begin{tcolorbox}[breakable, size=fbox, boxrule=1pt, pad at break*=1mm,colback=cellbackground, colframe=cellborder]
\prompt{In}{incolor}{22}{\boxspacing}
\begin{Verbatim}[commandchars=\\\{\}]
\PY{c+c1}{\PYZsh{} The output of this line is intended to produce a KeyError}
\PY{n+nb}{print}\PY{p}{(}\PY{n}{dictionary}\PY{p}{[}\PY{l+s+s1}{\PYZsq{}}\PY{l+s+s1}{key8}\PY{l+s+s1}{\PYZsq{}}\PY{p}{]}\PY{p}{)}
\end{Verbatim}
\end{tcolorbox}

    \begin{Verbatim}[commandchars=\\\{\}]

        ---------------------------------------------------------------------------

        KeyError                                  Traceback (most recent call last)

        <ipython-input-22-8d63520997fb> in <module>
          1 \# The output of this line is intended to produce a KeyError
    ----> 2 print(dictionary['key8'])
    

        KeyError: 'key8'

    \end{Verbatim}

    When using a square bracket lookup, it is common to use an if-else block
to check for containment first (with the keyword \texttt{in}) before
getting the item. On the other hand, you can use the \texttt{.get()}
method if you want to set a default value when the key is not found.
Let's compare these in the cells below:


    \hypertarget{this-prints-a-value}{%
\subsection{This prints a value}\label{this-prints-a-value}}

if `key1' in dictionary: print(``item found:'', dictionary{[}`key1'{]})
else: print(`key1 is not defined')

\hypertarget{same-as-what-you-get-with-get}{%
\subsection{Same as what you get with
get}\label{same-as-what-you-get-with-get}}

print(``item found:'', dictionary.get(`key1', -1))

    \begin{tcolorbox}[breakable, size=fbox, boxrule=1pt, pad at break*=1mm,colback=cellbackground, colframe=cellborder]
\prompt{In}{incolor}{23}{\boxspacing}
\begin{Verbatim}[commandchars=\\\{\}]
\PY{c+c1}{\PYZsh{} This prints a message because the key is not found}
\PY{k}{if} \PY{l+s+s1}{\PYZsq{}}\PY{l+s+s1}{key7}\PY{l+s+s1}{\PYZsq{}} \PY{o+ow}{in} \PY{n}{dictionary}\PY{p}{:}
    \PY{n+nb}{print}\PY{p}{(}\PY{n}{dictionary}\PY{p}{[}\PY{l+s+s1}{\PYZsq{}}\PY{l+s+s1}{key7}\PY{l+s+s1}{\PYZsq{}}\PY{p}{]}\PY{p}{)}
\PY{k}{else}\PY{p}{:}
    \PY{n+nb}{print}\PY{p}{(}\PY{l+s+s1}{\PYZsq{}}\PY{l+s+s1}{key does not exist!}\PY{l+s+s1}{\PYZsq{}}\PY{p}{)}

\PY{c+c1}{\PYZsh{} This prints \PYZhy{}1 because the key is not found and we set the default to \PYZhy{}1}
\PY{n+nb}{print}\PY{p}{(}\PY{n}{dictionary}\PY{o}{.}\PY{n}{get}\PY{p}{(}\PY{l+s+s1}{\PYZsq{}}\PY{l+s+s1}{key7}\PY{l+s+s1}{\PYZsq{}}\PY{p}{,} \PY{o}{\PYZhy{}}\PY{l+m+mi}{1}\PY{p}{)}\PY{p}{)}
\end{Verbatim}
\end{tcolorbox}

    \begin{Verbatim}[commandchars=\\\{\}]
key does not exist!
-1
    \end{Verbatim}

    \hypertarget{word-frequency-dictionary}{%
\subsection{Word frequency dictionary}\label{word-frequency-dictionary}}

    Now that we know the building blocks, let's finally take a look at the
\textbf{build\_freqs()} function below. This is the function that
creates the dictionary containing the word counts from each corpus.

    \begin{tcolorbox}[breakable, size=fbox, boxrule=1pt, pad at break*=1mm,colback=cellbackground, colframe=cellborder]
\prompt{In}{incolor}{24}{\boxspacing}
\begin{Verbatim}[commandchars=\\\{\}]
\PY{k}{def} \PY{n+nf}{build\PYZus{}freqs}\PY{p}{(}\PY{n}{tweets}\PY{p}{,} \PY{n}{ys}\PY{p}{)}\PY{p}{:}
    \PY{l+s+sd}{\PYZdq{}\PYZdq{}\PYZdq{}Build frequencies.}
\PY{l+s+sd}{    Input:}
\PY{l+s+sd}{        tweets: a list of tweets}
\PY{l+s+sd}{        ys: an m x 1 array with the sentiment label of each tweet}
\PY{l+s+sd}{            (either 0 or 1)}
\PY{l+s+sd}{    Output:}
\PY{l+s+sd}{        freqs: a dictionary mapping each (word, sentiment) pair to its}
\PY{l+s+sd}{        frequency}
\PY{l+s+sd}{    \PYZdq{}\PYZdq{}\PYZdq{}}
    \PY{c+c1}{\PYZsh{} Convert np array to list since zip needs an iterable.}
    \PY{c+c1}{\PYZsh{} The squeeze is necessary or the list ends up with one element.}
    \PY{c+c1}{\PYZsh{} Also note that this is just a NOP if ys is already a list.}
    \PY{n}{yslist} \PY{o}{=} \PY{n}{np}\PY{o}{.}\PY{n}{squeeze}\PY{p}{(}\PY{n}{ys}\PY{p}{)}\PY{o}{.}\PY{n}{tolist}\PY{p}{(}\PY{p}{)}

    \PY{c+c1}{\PYZsh{} Start with an empty dictionary and populate it by looping over all tweets}
    \PY{c+c1}{\PYZsh{} and over all processed words in each tweet.}
    \PY{n}{freqs} \PY{o}{=} \PY{p}{\PYZob{}}\PY{p}{\PYZcb{}}
    \PY{k}{for} \PY{n}{y}\PY{p}{,} \PY{n}{tweet} \PY{o+ow}{in} \PY{n+nb}{zip}\PY{p}{(}\PY{n}{yslist}\PY{p}{,} \PY{n}{tweets}\PY{p}{)}\PY{p}{:}
        \PY{k}{for} \PY{n}{word} \PY{o+ow}{in} \PY{n}{process\PYZus{}tweet}\PY{p}{(}\PY{n}{tweet}\PY{p}{)}\PY{p}{:}
            \PY{n}{pair} \PY{o}{=} \PY{p}{(}\PY{n}{word}\PY{p}{,} \PY{n}{y}\PY{p}{)}
            \PY{k}{if} \PY{n}{pair} \PY{o+ow}{in} \PY{n}{freqs}\PY{p}{:}
                \PY{n}{freqs}\PY{p}{[}\PY{n}{pair}\PY{p}{]} \PY{o}{+}\PY{o}{=} \PY{l+m+mi}{1}
            \PY{k}{else}\PY{p}{:}
                \PY{n}{freqs}\PY{p}{[}\PY{n}{pair}\PY{p}{]} \PY{o}{=} \PY{l+m+mi}{1}

    \PY{k}{return} \PY{n}{freqs}
\end{Verbatim}
\end{tcolorbox}

    \begin{tcolorbox}[breakable, size=fbox, boxrule=1pt, pad at break*=1mm,colback=cellbackground, colframe=cellborder]
\prompt{In}{incolor}{25}{\boxspacing}
\begin{Verbatim}[commandchars=\\\{\}]
\PY{c+c1}{\PYZsh{} Call the build\PYZus{}freqs function to create frequency dictionary based on tweets and labels}
\PY{n}{freqs} \PY{o}{=} \PY{n}{build\PYZus{}freqs}\PY{p}{(}\PY{n}{tweets}\PY{p}{,} \PY{n}{labels}\PY{p}{)}

\PY{c+c1}{\PYZsh{} Display the data type of freqs}
\PY{n+nb}{print}\PY{p}{(}\PY{n}{f}\PY{l+s+s1}{\PYZsq{}}\PY{l+s+s1}{type(freqs) = }\PY{l+s+s1}{\PYZob{}}\PY{l+s+s1}{type(freqs)\PYZcb{}}\PY{l+s+s1}{\PYZsq{}}\PY{p}{)}

\PY{c+c1}{\PYZsh{} Display the length of the dictionary}
\PY{n+nb}{print}\PY{p}{(}\PY{n}{f}\PY{l+s+s1}{\PYZsq{}}\PY{l+s+s1}{len(freqs) = }\PY{l+s+s1}{\PYZob{}}\PY{l+s+s1}{len(freqs)\PYZcb{}}\PY{l+s+s1}{\PYZsq{}}\PY{p}{)}
\end{Verbatim}
\end{tcolorbox}

    \begin{Verbatim}[commandchars=\\\{\}]
type(freqs) = <class 'dict'>
len(freqs) = 13067
    \end{Verbatim}

    \begin{tcolorbox}[breakable, size=fbox, boxrule=1pt, pad at break*=1mm,colback=cellbackground, colframe=cellborder]
\prompt{In}{incolor}{26}{\boxspacing}
\begin{Verbatim}[commandchars=\\\{\}]
\PY{c+c1}{\PYZsh{} prinf all the key\PYZhy{}value pair of frequency dictionary}
\PY{n+nb}{print}\PY{p}{(}\PY{n}{freqs}\PY{p}{)}
\end{Verbatim}
\end{tcolorbox}

    \begin{Verbatim}[commandchars=\\\{\}]
\{('followfriday', 1.0): 25, ('top', 1.0): 32, ('engag', 1.0): 7, ('member',
1.0): 16, ('commun', 1.0): 33, ('week', 1.0): 83, (':)', 1.0): 3568, ('hey',
1.0): 76, ('jame', 1.0): 7, ('odd', 1.0): 2, (':/', 1.0): 5, ('pleas', 1.0): 97,
('call', 1.0): 37, ('contact', 1.0): 7, ('centr', 1.0): 2, ('02392441234', 1.0):
<snip>
('ban', 0.0): 1, ('failsatlif', 0.0): 1, ('press', 0.0): 1, ('duper', 0.0): 1,
('waaah', 0.0): 1, ('jaebum', 0.0): 1, ('ahmad', 0.0): 1, ('maslan', 0.0): 1,
('hull', 0.0): 1, ('misser', 0.0): 1\}
    \end{Verbatim}

    Unfortunately, this does not help much to understand the data. It would
be better to visualize this output to gain better insights.

    ~\\
    \noindent\rule{16.5cm}{0.4pt}
    ~\newpage

    \hypertarget{table-of-word-counts}{%
\section{Table of word counts}\label{table-of-word-counts}}

    We will select a set of words that we would like to visualize. It is
better to store this temporary information in a table that is very easy
to use later.

    \begin{tcolorbox}[breakable, size=fbox, boxrule=1pt, pad at break*=1mm,colback=cellbackground, colframe=cellborder]
\prompt{In}{incolor}{27}{\boxspacing}
\begin{Verbatim}[commandchars=\\\{\}]
\PY{c+c1}{\PYZsh{} select some words to appear in the report. we will assume that each word is unique (i.e. no duplicates)}
\PY{n}{keys} \PY{o}{=} \PY{p}{[}\PY{l+s+s1}{\PYZsq{}}\PY{l+s+s1}{happi}\PY{l+s+s1}{\PYZsq{}}\PY{p}{,} \PY{l+s+s1}{\PYZsq{}}\PY{l+s+s1}{merri}\PY{l+s+s1}{\PYZsq{}}\PY{p}{,} \PY{l+s+s1}{\PYZsq{}}\PY{l+s+s1}{nice}\PY{l+s+s1}{\PYZsq{}}\PY{p}{,} \PY{l+s+s1}{\PYZsq{}}\PY{l+s+s1}{good}\PY{l+s+s1}{\PYZsq{}}\PY{p}{,} \PY{l+s+s1}{\PYZsq{}}\PY{l+s+s1}{bad}\PY{l+s+s1}{\PYZsq{}}\PY{p}{,} \PY{l+s+s1}{\PYZsq{}}\PY{l+s+s1}{sad}\PY{l+s+s1}{\PYZsq{}}\PY{p}{,} \PY{l+s+s1}{\PYZsq{}}\PY{l+s+s1}{mad}\PY{l+s+s1}{\PYZsq{}}\PY{p}{,} \PY{l+s+s1}{\PYZsq{}}\PY{l+s+s1}{best}\PY{l+s+s1}{\PYZsq{}}\PY{p}{,} \PY{l+s+s1}{\PYZsq{}}\PY{l+s+s1}{pretti}\PY{l+s+s1}{\PYZsq{}}\PY{p}{,}
        \PY{l+s+s1}{\PYZsq{}}\PY{l+s+s1}{❤}\PY{l+s+s1}{\PYZsq{}}\PY{p}{,} \PY{l+s+s1}{\PYZsq{}}\PY{l+s+s1}{:)}\PY{l+s+s1}{\PYZsq{}}\PY{p}{,} \PY{l+s+s1}{\PYZsq{}}\PY{l+s+s1}{:(}\PY{l+s+s1}{\PYZsq{}}\PY{p}{,} \PY{l+s+s1}{\PYZsq{}}\PY{l+s+s1}{��}\PY{l+s+s1}{\PYZsq{}}\PY{p}{,} \PY{l+s+s1}{\PYZsq{}}\PY{l+s+s1}{��}\PY{l+s+s1}{\PYZsq{}}\PY{p}{,} \PY{l+s+s1}{\PYZsq{}}\PY{l+s+s1}{��}\PY{l+s+s1}{\PYZsq{}}\PY{p}{,} \PY{l+s+s1}{\PYZsq{}}\PY{l+s+s1}{��}\PY{l+s+s1}{\PYZsq{}}\PY{p}{,} \PY{l+s+s1}{\PYZsq{}}\PY{l+s+s1}{♛}\PY{l+s+s1}{\PYZsq{}}\PY{p}{,}
        \PY{l+s+s1}{\PYZsq{}}\PY{l+s+s1}{song}\PY{l+s+s1}{\PYZsq{}}\PY{p}{,} \PY{l+s+s1}{\PYZsq{}}\PY{l+s+s1}{idea}\PY{l+s+s1}{\PYZsq{}}\PY{p}{,} \PY{l+s+s1}{\PYZsq{}}\PY{l+s+s1}{power}\PY{l+s+s1}{\PYZsq{}}\PY{p}{,} \PY{l+s+s1}{\PYZsq{}}\PY{l+s+s1}{play}\PY{l+s+s1}{\PYZsq{}}\PY{p}{,} \PY{l+s+s1}{\PYZsq{}}\PY{l+s+s1}{magnific}\PY{l+s+s1}{\PYZsq{}}\PY{p}{]}



\PY{c+c1}{\PYZsh{}each element consist of a sublist with this pattern: [\PYZlt{}word\PYZgt{}, \PYZlt{}positive\PYZus{}count\PYZgt{}, \PYZlt{}negative\PYZus{}count\PYZgt{}].}
\PY{n}{data} \PY{o}{=} \PY{p}{[}\PY{p}{]}

\PY{c+c1}{\PYZsh{} Iterate over each word in keys}
\PY{k}{for} \PY{n}{word} \PY{o+ow}{in} \PY{n}{keys}\PY{p}{:}
    
    \PY{c+c1}{\PYZsh{} initialize positive and negative counts}
    \PY{n}{pos} \PY{o}{=} \PY{l+m+mi}{0}
    \PY{n}{neg} \PY{o}{=} \PY{l+m+mi}{0}
    
    \PY{c+c1}{\PYZsh{} retrieve number of positive counts}
    \PY{k}{if} \PY{p}{(}\PY{n}{word}\PY{p}{,} \PY{l+m+mi}{1}\PY{p}{)} \PY{o+ow}{in} \PY{n}{freqs}\PY{p}{:}
        \PY{n}{pos} \PY{o}{=} \PY{n}{freqs}\PY{p}{[}\PY{p}{(}\PY{n}{word}\PY{p}{,} \PY{l+m+mi}{1}\PY{p}{)}\PY{p}{]}
        
    \PY{c+c1}{\PYZsh{} retrieve number of negative counts}
    \PY{k}{if} \PY{p}{(}\PY{n}{word}\PY{p}{,} \PY{l+m+mi}{0}\PY{p}{)} \PY{o+ow}{in} \PY{n}{freqs}\PY{p}{:}
        \PY{n}{neg} \PY{o}{=} \PY{n}{freqs}\PY{p}{[}\PY{p}{(}\PY{n}{word}\PY{p}{,} \PY{l+m+mi}{0}\PY{p}{)}\PY{p}{]}
        
    \PY{c+c1}{\PYZsh{} append the word counts to the table}
    \PY{n}{data}\PY{o}{.}\PY{n}{append}\PY{p}{(}\PY{p}{[}\PY{n}{word}\PY{p}{,} \PY{n}{pos}\PY{p}{,} \PY{n}{neg}\PY{p}{]}\PY{p}{)}
    
\PY{n}{data}
\end{Verbatim}
\end{tcolorbox}

            \begin{tcolorbox}[breakable, size=fbox, boxrule=.5pt, pad at break*=1mm, opacityfill=0]
\prompt{Out}{outcolor}{27}{\boxspacing}
\begin{Verbatim}[commandchars=\\\{\}]
[['happi', 211, 25],
 ['merri', 1, 0],
 ['nice', 98, 19],
 ['good', 238, 101],
 ['bad', 18, 73],
 ['sad', 5, 123],
 ['mad', 4, 11],
 ['best', 65, 22],
 ['pretti', 20, 15],
 ['❤', 29, 21],
 [':)', 3568, 2],
 [':(', 1, 4571],
 ['��', 1, 3],
 ['��', 0, 2],
 ['��', 5, 1],
 ['��', 2, 1],
 ['♛', 0, 210],
 ['song', 22, 27],
 ['idea', 26, 10],
 ['power', 7, 6],
 ['play', 46, 48],
 ['magnific', 2, 0]]
\end{Verbatim}
\end{tcolorbox}
        
    We can then use a scatter plot to inspect this table visually. Instead
of plotting the raw counts, we will plot it in the logarithmic scale to
take into account the wide discrepancies between the raw counts
(e.g.~\texttt{:)} has 3568 counts in the positive while only 2 in the
negative). The red line marks the boundary between positive and negative
areas. Words close to the red line can be classified as neutral.

    \begin{tcolorbox}[breakable, size=fbox, boxrule=1pt, pad at break*=1mm,colback=cellbackground, colframe=cellborder]
\prompt{In}{incolor}{28}{\boxspacing}
\begin{Verbatim}[commandchars=\\\{\}]
\PY{n}{fig}\PY{p}{,} \PY{n}{ax} \PY{o}{=} \PY{n}{plt}\PY{o}{.}\PY{n}{subplots}\PY{p}{(}\PY{n}{figsize} \PY{o}{=} \PY{p}{(}\PY{l+m+mi}{8}\PY{p}{,} \PY{l+m+mi}{8}\PY{p}{)}\PY{p}{)}

\PY{c+c1}{\PYZsh{} convert positive raw counts to logarithmic scale. we add 1 to avoid log(0)}
\PY{n}{x} \PY{o}{=} \PY{n}{np}\PY{o}{.}\PY{n}{log}\PY{p}{(}\PY{p}{[}\PY{n}{x}\PY{p}{[}\PY{l+m+mi}{1}\PY{p}{]} \PY{o}{+} \PY{l+m+mi}{1} \PY{k}{for} \PY{n}{x} \PY{o+ow}{in} \PY{n}{data}\PY{p}{]}\PY{p}{)}  

\PY{c+c1}{\PYZsh{} do the same for the negative counts}
\PY{n}{y} \PY{o}{=} \PY{n}{np}\PY{o}{.}\PY{n}{log}\PY{p}{(}\PY{p}{[}\PY{n}{x}\PY{p}{[}\PY{l+m+mi}{2}\PY{p}{]} \PY{o}{+} \PY{l+m+mi}{1} \PY{k}{for} \PY{n}{x} \PY{o+ow}{in} \PY{n}{data}\PY{p}{]}\PY{p}{)} 

\PY{c+c1}{\PYZsh{} Plot a dot for each pair of words}
\PY{n}{ax}\PY{o}{.}\PY{n}{scatter}\PY{p}{(}\PY{n}{x}\PY{p}{,} \PY{n}{y}\PY{p}{)}  

\PY{c+c1}{\PYZsh{} assign axis labels}
\PY{n}{plt}\PY{o}{.}\PY{n}{xlabel}\PY{p}{(}\PY{l+s+s2}{\PYZdq{}}\PY{l+s+s2}{Log Positive count}\PY{l+s+s2}{\PYZdq{}}\PY{p}{)}
\PY{n}{plt}\PY{o}{.}\PY{n}{ylabel}\PY{p}{(}\PY{l+s+s2}{\PYZdq{}}\PY{l+s+s2}{Log Negative count}\PY{l+s+s2}{\PYZdq{}}\PY{p}{)}

\PY{c+c1}{\PYZsh{} Add the word as the label at the same position as you added the points just before}
\PY{k}{for} \PY{n}{i} \PY{o+ow}{in} \PY{n+nb}{range}\PY{p}{(}\PY{l+m+mi}{0}\PY{p}{,} \PY{n+nb}{len}\PY{p}{(}\PY{n}{data}\PY{p}{)}\PY{p}{)}\PY{p}{:}
    \PY{n}{ax}\PY{o}{.}\PY{n}{annotate}\PY{p}{(}\PY{n}{data}\PY{p}{[}\PY{n}{i}\PY{p}{]}\PY{p}{[}\PY{l+m+mi}{0}\PY{p}{]}\PY{p}{,} \PY{p}{(}\PY{n}{x}\PY{p}{[}\PY{n}{i}\PY{p}{]}\PY{p}{,} \PY{n}{y}\PY{p}{[}\PY{n}{i}\PY{p}{]}\PY{p}{)}\PY{p}{,} \PY{n}{fontsize}\PY{o}{=}\PY{l+m+mi}{12}\PY{p}{)}

\PY{n}{ax}\PY{o}{.}\PY{n}{plot}\PY{p}{(}\PY{p}{[}\PY{l+m+mi}{0}\PY{p}{,} \PY{l+m+mi}{9}\PY{p}{]}\PY{p}{,} \PY{p}{[}\PY{l+m+mi}{0}\PY{p}{,} \PY{l+m+mi}{9}\PY{p}{]}\PY{p}{,} \PY{n}{color} \PY{o}{=} \PY{l+s+s1}{\PYZsq{}}\PY{l+s+s1}{red}\PY{l+s+s1}{\PYZsq{}}\PY{p}{)} \PY{c+c1}{\PYZsh{} Plot the red line that divides the 2 areas.}
\PY{n}{plt}\PY{o}{.}\PY{n}{show}\PY{p}{(}\PY{p}{)}
\end{Verbatim}
\end{tcolorbox}

    \begin{Verbatim}[commandchars=\\\{\}]
/usr/lib/python3/dist-packages/matplotlib/backends/backend\_agg.py:238:
RuntimeWarning: Glyph 128556 missing from current font.
  font.set\_text(s, 0.0, flags=flags)
/usr/lib/python3/dist-packages/matplotlib/backends/backend\_agg.py:201:
RuntimeWarning: Glyph 128556 missing from current font.
  font.set\_text(s, 0, flags=flags)
    \end{Verbatim}

    \begin{center}
    \adjustimage{max size={0.9\linewidth}{0.9\paperheight}}{assets/output_62_1.png}
    \end{center}
    { \hspace*{\fill} \\}
    
    This chart is straightforward to interpret. It shows that emoticons
\texttt{:)} and \texttt{:(} are very important for sentiment analysis.
Thus, we should not let preprocessing steps get rid of these symbols!

Furthermore, what is the meaning of the crown symbol? It seems to be
very negative!

    ~\\
    \noindent\rule{16.5cm}{0.4pt}

    \hypertarget{create-a-features-for-the-twitter-dataset-and-apply-any-machine-learning-algorithm-to-classify-the-tweets-and-check-the-accuracy-of-your-model.-try-with-logistic-regression-for-classification}{%
\section{Create a features for the twitter dataset and Apply any
machine learning algorithm to classify the tweets and check the accuracy
of your model. Try with Logistic Regression for
Classification}\label{create-a-features-for-the-twitter-dataset-and-apply-any-machine-learning-algorithm-to-classify-the-tweets-and-check-the-accuracy-of-your-model.-try-with-logistic-regression-for-classification}}

    \begin{tcolorbox}[breakable, size=fbox, boxrule=1pt, pad at break*=1mm,colback=cellbackground, colframe=cellborder]
\prompt{In}{incolor}{29}{\boxspacing}
\begin{Verbatim}[commandchars=\\\{\}]
\PY{k+kn}{import} \PY{n+nn}{pandas} \PY{k}{as} \PY{n+nn}{pd}
\PY{n}{data} \PY{o}{=} \PY{p}{[}\PY{p}{]}

\PY{k}{for} \PY{n}{i} \PY{o+ow}{in} \PY{n}{all\PYZus{}positive\PYZus{}tweets}\PY{p}{:}
  \PY{n}{data}\PY{o}{.}\PY{n}{append}\PY{p}{(}\PY{p}{[}\PY{n}{i}\PY{p}{,}\PY{l+m+mi}{1}\PY{p}{]}\PY{p}{)}
\PY{k}{for} \PY{n}{i} \PY{o+ow}{in} \PY{n}{all\PYZus{}negative\PYZus{}tweets}\PY{p}{:}
   \PY{n}{data}\PY{o}{.}\PY{n}{append}\PY{p}{(}\PY{p}{[}\PY{n}{i}\PY{p}{,}\PY{l+m+mi}{0}\PY{p}{]}\PY{p}{)}
\end{Verbatim}
\end{tcolorbox}

    \begin{tcolorbox}[breakable, size=fbox, boxrule=1pt, pad at break*=1mm,colback=cellbackground, colframe=cellborder]
\prompt{In}{incolor}{30}{\boxspacing}
\begin{Verbatim}[commandchars=\\\{\}]
\PY{n}{data} \PY{o}{=} \PY{n}{pd}\PY{o}{.}\PY{n}{DataFrame}\PY{p}{(}\PY{n}{data}\PY{p}{,}\PY{n}{columns} \PY{o}{=} \PY{p}{[}\PY{l+s+s1}{\PYZsq{}}\PY{l+s+s1}{tweet}\PY{l+s+s1}{\PYZsq{}}\PY{p}{,}\PY{l+s+s1}{\PYZsq{}}\PY{l+s+s1}{label}\PY{l+s+s1}{\PYZsq{}}\PY{p}{]}\PY{p}{)}
\PY{n+nb}{print}\PY{p}{(}\PY{n}{data}\PY{o}{.}\PY{n}{head}\PY{p}{(}\PY{p}{)}\PY{p}{)}
\end{Verbatim}
\end{tcolorbox}

    \begin{Verbatim}[commandchars=\\\{\}]
                                               tweet  label
0  \#FollowFriday @France\_Inte @PKuchly57 @Milipol{\ldots}      1
1  @Lamb2ja Hey James! How odd :/ Please call our{\ldots}      1
2  @DespiteOfficial we had a listen last night :){\ldots}      1
3                               @97sides CONGRATS :)      1
4  yeaaaah yippppy!!!  my accnt verified rqst has{\ldots}      1
    \end{Verbatim}

    \begin{tcolorbox}[breakable, size=fbox, boxrule=1pt, pad at break*=1mm,colback=cellbackground, colframe=cellborder]
\prompt{In}{incolor}{31}{\boxspacing}
\begin{Verbatim}[commandchars=\\\{\}]
\PY{k+kn}{from} \PY{n+nn}{sklearn}\PY{n+nn}{.}\PY{n+nn}{model\PYZus{}selection} \PY{k}{import} \PY{n}{train\PYZus{}test\PYZus{}split}
\PY{k+kn}{from} \PY{n+nn}{sklearn}\PY{n+nn}{.}\PY{n+nn}{feature\PYZus{}extraction}\PY{n+nn}{.}\PY{n+nn}{text} \PY{k}{import} \PY{n}{TfidfVectorizer}\PY{p}{,}\PY{n}{CountVectorizer}
\PY{n}{x} \PY{o}{=} \PY{n}{data}\PY{p}{[}\PY{l+s+s1}{\PYZsq{}}\PY{l+s+s1}{tweet}\PY{l+s+s1}{\PYZsq{}}\PY{p}{]}
\PY{n}{y} \PY{o}{=} \PY{n}{data}\PY{p}{[}\PY{l+s+s1}{\PYZsq{}}\PY{l+s+s1}{label}\PY{l+s+s1}{\PYZsq{}}\PY{p}{]}
\end{Verbatim}
\end{tcolorbox}

    \begin{tcolorbox}[breakable, size=fbox, boxrule=1pt, pad at break*=1mm,colback=cellbackground, colframe=cellborder]
\prompt{In}{incolor}{32}{\boxspacing}
\begin{Verbatim}[commandchars=\\\{\}]
\PY{n}{x\PYZus{}train}\PY{p}{,}\PY{n}{x\PYZus{}test}\PY{p}{,}\PY{n}{y\PYZus{}train}\PY{p}{,}\PY{n}{y\PYZus{}test} \PY{o}{=} \PY{n}{train\PYZus{}test\PYZus{}split}\PY{p}{(}\PY{n}{x}\PY{p}{,}\PY{n}{y}\PY{p}{,}\PY{n}{test\PYZus{}size}\PY{o}{=}\PY{l+m+mf}{0.3}\PY{p}{,}\PY{n}{random\PYZus{}state} \PY{o}{=} \PY{l+m+mi}{53}\PY{p}{)}
\end{Verbatim}
\end{tcolorbox}

    \begin{tcolorbox}[breakable, size=fbox, boxrule=1pt, pad at break*=1mm,colback=cellbackground, colframe=cellborder]
\prompt{In}{incolor}{34}{\boxspacing}
\begin{Verbatim}[commandchars=\\\{\}]
\PY{n}{count\PYZus{}vectorizer} \PY{o}{=} \PY{n}{CountVectorizer}\PY{p}{(}\PY{n}{stop\PYZus{}words}\PY{o}{=}\PY{l+s+s1}{\PYZsq{}}\PY{l+s+s1}{english}\PY{l+s+s1}{\PYZsq{}}\PY{p}{,}\PY{n}{min\PYZus{}df}\PY{o}{=}\PY{l+m+mi}{6}\PY{p}{)}
\PY{n}{count\PYZus{}train} \PY{o}{=} \PY{n}{count\PYZus{}vectorizer}\PY{o}{.}\PY{n}{fit\PYZus{}transform}\PY{p}{(}\PY{n}{x\PYZus{}train}\PY{p}{)}
\PY{n}{count\PYZus{}test} \PY{o}{=} \PY{n}{count\PYZus{}vectorizer}\PY{o}{.}\PY{n}{transform}\PY{p}{(}\PY{n}{x\PYZus{}test}\PY{p}{)}
\PY{n+nb}{print}\PY{p}{(}\PY{n}{count\PYZus{}vectorizer}\PY{o}{.}\PY{n}{get\PYZus{}feature\PYZus{}names}\PY{p}{(}\PY{p}{)}\PY{p}{[}\PY{p}{:}\PY{l+m+mi}{10}\PY{p}{]}\PY{p}{)}
\PY{n+nb}{print}\PY{p}{(}\PY{n}{count\PYZus{}vectorizer}\PY{o}{.}\PY{n}{vocabulary\PYZus{}}\PY{o}{.}\PY{n}{keys}\PY{p}{(}\PY{p}{)}\PY{p}{)}
\end{Verbatim}
\end{tcolorbox}

    \begin{Verbatim}[commandchars=\\\{\}]
['000', '07', '10', '100', '11', '12', '13', '15', '20', '2015']
dict\_keys(['happy', 'friday', 'just', 'enjoyed', 'super', '100', 'pleasure',
'let', 'know', 'need', 'http', 'nice', 'day', 'queen', 'hi', 'ask', 'fingers',
'cut', 'food', 'shop', 'week', 'https', '5sos', 'shot', 'listening', 'days',
'think', 'love', 'meet', 'person', 'guess', 'spent', 'half', 'outside', 'bc',
'left', 'work', 'hope', 'needs', 'leave', 'soon', 'party', 'cheers', 'cause',
'haha', 'maybe', 'thanks', 'rest', 'sounds', 'lovely', 'don', 'plans', 'add',
'new', 'moment', 'inside', 'today', 'asked', 'members', 'favourite', 'brain',
'thought', 'friend', 'lot', 'makes', 'followed', 'justinbieber', 'agree',
'phone', 'saturday', 'fuck', 'ill', 'promise', 'dog', 'stress', 'haven', 'seen',
'years', 'oh', 'sorry', 'hear', 'having', 'tuesday', 'want', 'finally', 'bed',
'good', 'night', 'hurt', 'allah', 'yes', 'better', 'plan', 'look', 'forward',
'school', 'weeks', 'feeling', 'saw', 'follow', 'like', 'dark', 'plz', 'far',
'fan', 'wanna', 'free', 'check', 'thx', 'drive', 'stats', 'arrived', 'follower',
'unfollowers', 'going', 'spend', 'leaving', 'tomorrow', 'ok', 'chill', 'aw',
'babe', 'england', 'll', 'turn', 'welcome', 'amp', 'sharing', 'enjoy', 'app',
'wishing', 'weekend', 'ya', 'green', 'tea', 'bored', 'did', 'crying', 'red',
'train', 'retweet', 'active', 'followers', 'getting', 'sick', 'late', 'event',
'sure', 'stuff', 'august', 'thank', 'sleep', 'sweet', 'dreams',
'zayniscomingbackonjuly26', 'im', 'zayn', 'come', 'believe', 'omg', 'birthday',
'missed', 'fun', 'home', 'face', 'couldn', 'kid', 'right', 'literally',
'people', 'actually', 'doing', 'thing', 'rock', 'tired', 'head', 'afternoon',
'lets', 'read', 'al', 'finish', 'man', 'long', 'does', 'wish', 'year', 've',
'watching', 'gutted', 'miss', 'lady', 'talk', 'help', 'watch',
'bajrangibhaijaanhighestweek1', 'wait', 'really', 'told', 'ran', 'away',
'forever', 'beat', 'visit', 'na', 'aren', 'say', 'proud', 'heart', 'broken',
'works', 'smile', 'mind', 'following', 'body', 'wonderful', 'way', 'guys',
'video', 'amazing', 'ohh', 'team', 'thats', 'question', 'happened', 'hey',
'text', 'july', '24', '2015', '07', 'yeah', 'fucking', 'email', 'information',
'pro', 'instead', 'set', 'isn', 'black', 'cat', 'bad', 'luck', 'kinda', 'ago',
'got', 'sent', 'songs', 'gt', 'stefaniescott', 'real', 'favorite', 'sun',
'beli', 'eve', 'wi', 'justi', 'x15', 'see', 'me', 'funny', 'dress', 'pretty',
'change', 'close', 'xd', 'feel', 'shit', 'morning', 'big', 'cool', 'dream',
'fab', 'release', 'win', 'huge', 'giveaway', 'running', 'sore', 'throat',
'fever', 'create', 'fucked', 'using', 'appreciate', 'support', 'vote', 'games',
'try', 'best', 'making', 'sad', 'glad', 'liked', 'sponsor', 'feels', 'shift',
'chocolate', 'xx', 'broke', 'money', 'dm', 'make', 'fast', 'received', 'wrong',
'order', 'sort', 'uk', 'okay', 'la', 'wont', 'thinking', 'ur', 'hate', 'shame',
'tweets', 'waiting', 'invite', 'working', 'hahaha', 'forgotten', 'sooo',
'jealous', 'heard', 'eat', 'weird', 'fine', 'game', 'success', 'ang', 'cold',
'store', 'time', 'doesn', 'update', 'mention', 'longer', 'awful', 'story',
'lost', 'id', 'airport', 'concert', 'god', 'damn', 'gonna', 'didnt', 'congrats',
'infinite', 'fav', 'song', 'rip', 'looks', 'awesome', 'cars', 'house', 'yay',
'isnt', 'loved', 'friends', 'rn', 'thankyou', 'sa', 'sir', 'group', 'facebook',
'small', 'lt', 'harry', 'pls', 'bae', 'ignore', 'taking', 'reply', 'answer',
'pizza', 'wanted', '40', 'goodmorning', 'life', 'easy', 'account',
'bhaktisbanter', 'flipkartfashionfriday', 'worst', 'pain', 'bam',
'barsandmelody', 'bestfriend', '969horan696', 'loves', 'warsaw', 'hello',
'youth', 'job', 'opportunities', 'tolajobjobs', 'channel', 'worse', 'awake',
'yep', 'weather', 'bag', 'idk', 'play', 'talking', 'tweet', 'dont', 'hard',
'info', 'dying', 'little', 'coming', 'ff', 'old', 'pic', 'used', 'buy', 'meant',
'rude', 'mean', 'usually', 'use', 'dude', 'waste', 'twitter', 'tbh', 'blue',
'fb', '15', 'later', '10', 'snapchat', 'kik', 'chat', 'xxx', 'kikmeboys',
'travel', 'box', 'bring', 'beautiful', 'place', 'picture', 'won', 'able', '50',
'dead', '300', 'send', 'john', 'checked', 'tell', '20', 'mins', 'fair', 'cute',
'saying', 'short', 'finished', 'spain', 'excited', 'season', 'jnlazts',
'rcvcyyo0iq', 'youtube', 'link', 'trying', 'live', 'srsly', 'seeing',
'brilliant', 'gorgeous', 'shout', 'girl', 'came', 'went', 'college', 'baby',
'stop', 'playing', 'feelings', 'aint', 'hoping', 'tgif', 'tonight', 'view',
'eyes', 'notice', 'great', 'start', 'light', 'business', 'jaymcguiness',
'family', 'trip', 'early', 'paper', 'sigh', 'likeforlike', 'write', 'possible',
'months', '2nd', 'choice', 'iphone', 'facetime', 'pictures', 'cuz', 'sucks',
'badly', 'second', 'smiling', 'listen', 'fback', 'hugs', 'didn', 'definitely',
'took', 'followfriday', 'supports', 'community', 'photo', 'happens', 'kidding',
'littlemix', 'girls', 'says', 'available', 'awwww', 'forget', 'music',
'looking', 'kikgirl', 'french', 'model', 'watched', 'wow', 'scared', 'af',
'course', 'waking', 'catch', 'barely', 'dinner', 'date', 'luke', 'reason',
'youre', 'whats', 'ice', 'cream', 'missing', 'needed', 'uniteblue', 'tcot',
'sadly', 'remember', 'lol', 'hug', 'class', 'women', 'art', 'everyday',
'breakfast', 'zaynmalik', 'followback', 'meeting', 'gets', 'amber', 'post',
'flight', 'tried', 'btw', 'quite', 'wasn', 'men', 'tour', 'said', 'board',
'bye', 'sound', 'decide', 'book', 'hot', 'kiksex', 'tagsforlikes', 'reading',
'interesting', 'kind', 'totally', 'stay', 'impastel', 'city', 'country', 'ones',
'met', 'stuck', 'certain', 'case', 'message', 'easier', 'craving', 'lunch',
'started', 'things', 'busy', 'truth', 'die', 'save', 'bday', 'boring',
'original', 'door', 'guy', 'poor', 'english', 'worry', 'bit', 'hair', 'pass',
'words', 'past', 'news', 'wforwoman', 'wsalelove', 'sign', 'wouldn', 'end',
'yup', '30', 'lucky', 'nope', 'air', 'line', 'care', 'ha', 'touch', 'forgot',
'gold', 'stream', 'keeps', 'bath', 'goodnight', 'park', 'hotel', 'hopefully',
'hours', 'ahh', 'ate', 'unfortunately', 'calm', 'tummy', 'hurts', 'bro',
'coffee', 'special', 'address', 'problem', 'power', 'festival', 'ive', 'times',
'world', 'gift', 'true', 'bank', 'cake', 'car', 'regret', 'probably', 'fix',
'plays', 'windows', 'cheese', 'white', 'near', 'exactly', 'list', 'ty', 'mum',
'london', 'lonely', 'issue', 'future', 'starts', 'cutie', 'congratulations',
'hornykik', 'internet', 'means', 'aww', 'stomach', 'number', 'woke', 'shopping',
'parents', 'vacation', 'ah', 'anybody', 'added', 'final', 'design',
'ext098yq1b', 'books', 'hun', 'ball', 'questions', 'episode', 'worth', 'learn',
'lots', 'hehe', 'high', 'kids', 'wonder', 'boy', 'wants', 'rain', 'breaking',
'chris', 'selfie', 'alice', 'pics', 'emilybett', 'teenchoice',
'choiceinternationalartist', 'superjunior', 'bet', 'site', '000', 'pool',
'awww', 'album', 'ready', 'goes', 'da', 'wife', 'kikmenow', 'summer', 'likes',
'open', 'walk', 'hungry', 'ain', 'online', 'feedback', 'perfect', 'dots',
'braindots', 'hour', 'rafaelallmark', 'supporting', 'blog', 'interested',
'lmao', 'month', 'website', 'follback', 'tho', 'ass', 'fall', 'asleep',
'tweeting', 'uberuk', 'uber', 'anymore', 'acc', 'load', 'fantastic', 'sending',
'tl', 'chance', 'wake', '11', 'sunshine', 'deserve', 'wtf', 'driving', 'ugly',
'dear', 'towns', 'movies', 'tv', 'skype', 'shows', 'painful', 'training', 'hai',
'gotta', 'join', 'yesterday', 'gone', 'absolutely', 'ugh', 'ubericecream',
'click', 'aqui', 'film', 'cover', 'felt', 'louis', 'influencers', 'young',
'bravefrontiergl', 'hurry', 'mad', 'bb', 'ears', 'surprise', 'selenagomez',
'bitch', 'crazy', 'kikhorny', 'happiness', 'appreciated', 'havent', 'earlier',
'videos', 'share', 'boys', 'idea', 'low', 'moving', 'bby', 'especially',
'turned', 'middle', 'lil', 'giving', 'offer', 'oppa', 'single', 'download',
'lose', 'liam', 'card', 'wet', 'indiemusic', 'sexy', 'failed', 'till',
'mistake', 'understand', 'paid', 'pay', 'rt', 'tom', 'happen', 'different',
'voice', 'apparently', 'fans', 'mom', 'hit', 'run', 'contact', 'movie',
'adeccowaytowork', 'kunoriforceo', 'ceo1month', 'sister', 'vidcon', 'hahahaha',
'ended', 'shall', 'review', 'raining', 'beach', 'holiday', 'outfit', '13',
'fact', 'hold', 'kikme', 'real\_liam\_payne', 'fell', 'knows', 'sex', 'gave',
'alright', 'point', 'tickets', 'hell', 'asking', 'joined', 'keeping', 'office',
'count', 'points', 'collection', 'dad', 'computer', 'conversation', 'station',
'leeds', 'ye', 'se', 'ticket', 'garden', 'solo', 'match', 'mr', 'pick', 'area',
'monday', 'water', 'ko', 'sunday', 'choose', 'details', 'goodbye', 'matt',
'warm', 'couple', 'slept', 'mood', 'pa', 'seriously', 'regular', 'quick',
'comes', 'eating', 'sleeping', 'safe', 'stage', 'realized', 'unless',
'planning', 'completely', 'madrid', 'kit', 'instagram', 'stupid', 'huhu',
'secret', 'pre', 'schedule', 'page', 'photos', '12', 'horrible', 'mm', 'drop',
'writing', 'kiksexting', 'annoying', 'cc', 'article', 'carterreynolds', '33',
'entire', 'word', 'clothes', 'australia', 'til', 'looked', 'got7', 'brother',
'luv'])
    \end{Verbatim}

    \begin{tcolorbox}[breakable, size=fbox, boxrule=1pt, pad at break*=1mm,colback=cellbackground, colframe=cellborder]
\prompt{In}{incolor}{35}{\boxspacing}
\begin{Verbatim}[commandchars=\\\{\}]
\PY{n+nb}{print}\PY{p}{(}\PY{n}{count\PYZus{}vectorizer}\PY{o}{.}\PY{n}{get\PYZus{}params}\PY{p}{(}\PY{n}{deep}\PY{o}{=}\PY{k+kc}{True}\PY{p}{)}\PY{p}{)}
\end{Verbatim}
\end{tcolorbox}

    \begin{Verbatim}[commandchars=\\\{\}]
\{'analyzer': 'word', 'binary': False, 'decode\_error': 'strict', 'dtype': <class
'numpy.int64'>, 'encoding': 'utf-8', 'input': 'content', 'lowercase': True,
'max\_df': 1.0, 'max\_features': None, 'min\_df': 6, 'ngram\_range': (1, 1),
'preprocessor': None, 'stop\_words': 'english', 'strip\_accents': None,
'token\_pattern': '(?u)\textbackslash{}\textbackslash{}b\textbackslash{}\textbackslash{}w\textbackslash{}\textbackslash{}w+\textbackslash{}\textbackslash{}b', 'tokenizer': None, 'vocabulary': None\}
    \end{Verbatim}

    \begin{tcolorbox}[breakable, size=fbox, boxrule=1pt, pad at break*=1mm,colback=cellbackground, colframe=cellborder]
\prompt{In}{incolor}{36}{\boxspacing}
\begin{Verbatim}[commandchars=\\\{\}]
\PY{n}{count\PYZus{}df} \PY{o}{=} \PY{n}{pd}\PY{o}{.}\PY{n}{DataFrame}\PY{p}{(}\PY{n}{count\PYZus{}train}\PY{o}{.}\PY{n}{A}\PY{p}{,} \PY{n}{columns}\PY{o}{=}\PY{n}{count\PYZus{}vectorizer}\PY{o}{.}\PY{n}{get\PYZus{}feature\PYZus{}names}\PY{p}{(}\PY{p}{)}\PY{p}{)}
\PY{n+nb}{print}\PY{p}{(}\PY{n}{count\PYZus{}df}\PY{p}{)}
\end{Verbatim}
\end{tcolorbox}

    \begin{Verbatim}[commandchars=\\\{\}]
      000  07  10  100  11  12  13  15  20  2015  {\ldots}  young  youre  youth  \textbackslash{}
0       0   0   0    0   0   0   0   0   0     0  {\ldots}      0      0      0
1       0   0   0    1   0   0   0   0   0     0  {\ldots}      0      0      0
2       0   0   0    0   0   0   0   0   0     0  {\ldots}      0      0      0
3       0   0   0    0   0   0   0   0   0     0  {\ldots}      0      0      0
4       0   0   0    0   0   0   0   0   0     0  {\ldots}      0      0      0
{\ldots}   {\ldots}  ..  ..  {\ldots}  ..  ..  ..  ..  ..   {\ldots}  {\ldots}    {\ldots}    {\ldots}    {\ldots}
6995    0   0   0    0   0   0   0   0   0     0  {\ldots}      0      0      0
6996    0   0   0    0   0   0   0   0   0     0  {\ldots}      0      0      0
6997    0   0   0    0   0   0   0   0   0     0  {\ldots}      0      0      0
6998    0   0   0    0   0   0   0   0   0     0  {\ldots}      0      0      0
6999    0   0   0    0   0   0   0   0   0     0  {\ldots}      0      0      0

      youtube  yup  zayn  zayniscomingbackonjuly26  zaynmalik  me  see
0           0    0     0                         0          0   0    0
1           0    0     0                         0          0   0    0
2           0    0     0                         0          0   0    0
3           0    0     0                         0          0   0    0
4           0    0     0                         0          0   0    0
{\ldots}       {\ldots}  {\ldots}   {\ldots}                       {\ldots}        {\ldots}  ..  {\ldots}
6995        0    0     0                         0          0   0    0
6996        0    0     0                         0          0   0    0
6997        0    0     0                         0          0   0    0
6998        0    0     0                         0          0   0    0
6999        0    0     0                         0          0   0    0

[7000 rows x 959 columns]
    \end{Verbatim}

    \begin{tcolorbox}[breakable, size=fbox, boxrule=1pt, pad at break*=1mm,colback=cellbackground, colframe=cellborder]
\prompt{In}{incolor}{37}{\boxspacing}
\begin{Verbatim}[commandchars=\\\{\}]
\PY{k+kn}{from} \PY{n+nn}{sklearn}\PY{n+nn}{.}\PY{n+nn}{naive\PYZus{}bayes} \PY{k}{import} \PY{n}{MultinomialNB}
\PY{k+kn}{from} \PY{n+nn}{sklearn} \PY{k}{import} \PY{n}{metrics}

\PY{n}{clf} \PY{o}{=} \PY{n}{MultinomialNB}\PY{p}{(}\PY{p}{)}
\PY{n}{clf}\PY{o}{.}\PY{n}{fit}\PY{p}{(}\PY{n}{count\PYZus{}train}\PY{p}{,}\PY{n}{y\PYZus{}train}\PY{p}{)}
\PY{n}{pred} \PY{o}{=} \PY{n}{clf}\PY{o}{.}\PY{n}{predict}\PY{p}{(}\PY{n}{count\PYZus{}test}\PY{p}{)}
\end{Verbatim}
\end{tcolorbox}

    \begin{tcolorbox}[breakable, size=fbox, boxrule=1pt, pad at break*=1mm,colback=cellbackground, colframe=cellborder]
\prompt{In}{incolor}{38}{\boxspacing}
\begin{Verbatim}[commandchars=\\\{\}]
\PY{n}{score} \PY{o}{=} \PY{n}{metrics}\PY{o}{.}\PY{n}{accuracy\PYZus{}score}\PY{p}{(}\PY{n}{y\PYZus{}test}\PY{p}{,}\PY{n}{pred}\PY{p}{)}
\PY{n+nb}{print}\PY{p}{(}\PY{n}{score}\PY{p}{)}

\PY{n}{cm} \PY{o}{=} \PY{n}{metrics}\PY{o}{.}\PY{n}{confusion\PYZus{}matrix}\PY{p}{(}\PY{n}{y\PYZus{}test}\PY{p}{,}\PY{n}{pred}\PY{p}{,}\PY{n}{labels} \PY{o}{=} \PY{p}{[}\PY{l+m+mi}{1}\PY{p}{,}\PY{l+m+mi}{0}\PY{p}{]}\PY{p}{)}
\PY{n+nb}{print}\PY{p}{(}\PY{n}{cm}\PY{p}{)}
\end{Verbatim}
\end{tcolorbox}

    \begin{Verbatim}[commandchars=\\\{\}]
0.742
[[1069  466]
 [ 308 1157]]
    \end{Verbatim}

    \begin{tcolorbox}[breakable, size=fbox, boxrule=1pt, pad at break*=1mm,colback=cellbackground, colframe=cellborder]
\prompt{In}{incolor}{39}{\boxspacing}
\begin{Verbatim}[commandchars=\\\{\}]
\PY{k+kn}{from} \PY{n+nn}{sklearn}\PY{n+nn}{.}\PY{n+nn}{ensemble} \PY{k}{import} \PY{n}{RandomForestClassifier}
\PY{k+kn}{from} \PY{n+nn}{sklearn}\PY{n+nn}{.}\PY{n+nn}{svm} \PY{k}{import} \PY{n}{SVC}

\PY{n}{clf} \PY{o}{=} \PY{n}{SVC}\PY{p}{(}\PY{p}{)}
\end{Verbatim}
\end{tcolorbox}

    \begin{tcolorbox}[breakable, size=fbox, boxrule=1pt, pad at break*=1mm,colback=cellbackground, colframe=cellborder]
\prompt{In}{incolor}{40}{\boxspacing}
\begin{Verbatim}[commandchars=\\\{\}]
\PY{n}{clf}\PY{o}{.}\PY{n}{fit}\PY{p}{(}\PY{n}{count\PYZus{}train}\PY{p}{,}\PY{n}{y\PYZus{}train}\PY{p}{)}
\PY{n}{pred} \PY{o}{=} \PY{n}{clf}\PY{o}{.}\PY{n}{predict}\PY{p}{(}\PY{n}{count\PYZus{}test}\PY{p}{)}
\end{Verbatim}
\end{tcolorbox}

    \begin{tcolorbox}[breakable, size=fbox, boxrule=1pt, pad at break*=1mm,colback=cellbackground, colframe=cellborder]
\prompt{In}{incolor}{41}{\boxspacing}
\begin{Verbatim}[commandchars=\\\{\}]
\PY{n}{score} \PY{o}{=} \PY{n}{metrics}\PY{o}{.}\PY{n}{accuracy\PYZus{}score}\PY{p}{(}\PY{n}{y\PYZus{}test}\PY{p}{,}\PY{n}{pred}\PY{p}{)}
\PY{n+nb}{print}\PY{p}{(}\PY{n}{score}\PY{p}{)}

\PY{n}{cm} \PY{o}{=} \PY{n}{metrics}\PY{o}{.}\PY{n}{confusion\PYZus{}matrix}\PY{p}{(}\PY{n}{y\PYZus{}test}\PY{p}{,}\PY{n}{pred}\PY{p}{,}\PY{n}{labels} \PY{o}{=} \PY{p}{[}\PY{l+m+mi}{1}\PY{p}{,}\PY{l+m+mi}{0}\PY{p}{]}\PY{p}{)}
\PY{n+nb}{print}\PY{p}{(}\PY{n}{cm}\PY{p}{)}
\end{Verbatim}
\end{tcolorbox}

    \begin{Verbatim}[commandchars=\\\{\}]
0.7383333333333333
[[1030  505]
 [ 280 1185]]
    \end{Verbatim}

    ~\\
    \noindent\rule{16.5cm}{0.4pt}

\end{document}
