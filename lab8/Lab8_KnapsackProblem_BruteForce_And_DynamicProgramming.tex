\documentclass[11pt]{article}

    \usepackage[breakable]{tcolorbox}
    \usepackage{parskip} % Stop auto-indenting (to mimic markdown behaviour)
    
    \usepackage{iftex}
    \ifPDFTeX
    	\usepackage[T1]{fontenc}
    	\usepackage{mathpazo}
    \else
    	\usepackage{fontspec}
    \fi

    % Basic figure setup, for now with no caption control since it's done
    % automatically by Pandoc (which extracts ![](path) syntax from Markdown).
    \usepackage{graphicx}
    % Maintain compatibility with old templates. Remove in nbconvert 6.0
    \let\Oldincludegraphics\includegraphics
    % Ensure that by default, figures have no caption (until we provide a
    % proper Figure object with a Caption API and a way to capture that
    % in the conversion process - todo).
    \usepackage{caption}
    \DeclareCaptionFormat{nocaption}{}
    \captionsetup{format=nocaption,aboveskip=0pt,belowskip=0pt}

    \usepackage[Export]{adjustbox} % Used to constrain images to a maximum size
    \adjustboxset{max size={0.9\linewidth}{0.9\paperheight}}
    \usepackage{float}
    \floatplacement{figure}{H} % forces figures to be placed at the correct location
    \usepackage{xcolor} % Allow colors to be defined
    \usepackage{enumerate} % Needed for markdown enumerations to work
    \usepackage{geometry} % Used to adjust the document margins
    \usepackage{amsmath} % Equations
    \usepackage{amssymb} % Equations
    \usepackage{textcomp} % defines textquotesingle
    % Hack from http://tex.stackexchange.com/a/47451/13684:
    \AtBeginDocument{%
        \def\PYZsq{\textquotesingle}% Upright quotes in Pygmentized code
    }
    \usepackage{upquote} % Upright quotes for verbatim code
    \usepackage{eurosym} % defines \euro
    \usepackage[mathletters]{ucs} % Extended unicode (utf-8) support
    \usepackage{fancyvrb} % verbatim replacement that allows latex

    % The hyperref package gives us a pdf with properly built
    % internal navigation ('pdf bookmarks' for the table of contents,
    % internal cross-reference links, web links for URLs, etc.)
    \usepackage{hyperref}
    % The default LaTeX title has an obnoxious amount of whitespace. By default,
    % titling removes some of it. It also provides customization options.
    \usepackage{titling}
    \usepackage{longtable} % longtable support required by pandoc >1.10
    \usepackage{booktabs}  % table support for pandoc > 1.12.2
    \usepackage[inline]{enumitem} % IRkernel/repr support (it uses the enumerate* environment)
    \usepackage[normalem]{ulem} % ulem is needed to support strikethroughs (\sout)
                                % normalem makes italics be italics, not underlines
    \usepackage{mathrsfs}
    

    
    % Colors for the hyperref package
    \definecolor{urlcolor}{rgb}{0,.145,.698}
    \definecolor{linkcolor}{rgb}{.71,0.21,0.01}
    \definecolor{citecolor}{rgb}{.12,.54,.11}

    % ANSI colors
    \definecolor{ansi-black}{HTML}{3E424D}
    \definecolor{ansi-black-intense}{HTML}{282C36}
    \definecolor{ansi-red}{HTML}{E75C58}
    \definecolor{ansi-red-intense}{HTML}{B22B31}
    \definecolor{ansi-green}{HTML}{00A250}
    \definecolor{ansi-green-intense}{HTML}{007427}
    \definecolor{ansi-yellow}{HTML}{DDB62B}
    \definecolor{ansi-yellow-intense}{HTML}{B27D12}
    \definecolor{ansi-blue}{HTML}{208FFB}
    \definecolor{ansi-blue-intense}{HTML}{0065CA}
    \definecolor{ansi-magenta}{HTML}{D160C4}
    \definecolor{ansi-magenta-intense}{HTML}{A03196}
    \definecolor{ansi-cyan}{HTML}{60C6C8}
    \definecolor{ansi-cyan-intense}{HTML}{258F8F}
    \definecolor{ansi-white}{HTML}{C5C1B4}
    \definecolor{ansi-white-intense}{HTML}{A1A6B2}
    \definecolor{ansi-default-inverse-fg}{HTML}{FFFFFF}
    \definecolor{ansi-default-inverse-bg}{HTML}{000000}

    % commands and environments needed by pandoc snippets
    % extracted from the output of `pandoc -s`
    \providecommand{\tightlist}{%
      \setlength{\itemsep}{0pt}\setlength{\parskip}{0pt}}
    \DefineVerbatimEnvironment{Highlighting}{Verbatim}{commandchars=\\\{\}}
    % Add ',fontsize=\small' for more characters per line
    \newenvironment{Shaded}{}{}
    \newcommand{\KeywordTok}[1]{\textcolor[rgb]{0.00,0.44,0.13}{\textbf{{#1}}}}
    \newcommand{\DataTypeTok}[1]{\textcolor[rgb]{0.56,0.13,0.00}{{#1}}}
    \newcommand{\DecValTok}[1]{\textcolor[rgb]{0.25,0.63,0.44}{{#1}}}
    \newcommand{\BaseNTok}[1]{\textcolor[rgb]{0.25,0.63,0.44}{{#1}}}
    \newcommand{\FloatTok}[1]{\textcolor[rgb]{0.25,0.63,0.44}{{#1}}}
    \newcommand{\CharTok}[1]{\textcolor[rgb]{0.25,0.44,0.63}{{#1}}}
    \newcommand{\StringTok}[1]{\textcolor[rgb]{0.25,0.44,0.63}{{#1}}}
    \newcommand{\CommentTok}[1]{\textcolor[rgb]{0.38,0.63,0.69}{\textit{{#1}}}}
    \newcommand{\OtherTok}[1]{\textcolor[rgb]{0.00,0.44,0.13}{{#1}}}
    \newcommand{\AlertTok}[1]{\textcolor[rgb]{1.00,0.00,0.00}{\textbf{{#1}}}}
    \newcommand{\FunctionTok}[1]{\textcolor[rgb]{0.02,0.16,0.49}{{#1}}}
    \newcommand{\RegionMarkerTok}[1]{{#1}}
    \newcommand{\ErrorTok}[1]{\textcolor[rgb]{1.00,0.00,0.00}{\textbf{{#1}}}}
    \newcommand{\NormalTok}[1]{{#1}}
    
    % Additional commands for more recent versions of Pandoc
    \newcommand{\ConstantTok}[1]{\textcolor[rgb]{0.53,0.00,0.00}{{#1}}}
    \newcommand{\SpecialCharTok}[1]{\textcolor[rgb]{0.25,0.44,0.63}{{#1}}}
    \newcommand{\VerbatimStringTok}[1]{\textcolor[rgb]{0.25,0.44,0.63}{{#1}}}
    \newcommand{\SpecialStringTok}[1]{\textcolor[rgb]{0.73,0.40,0.53}{{#1}}}
    \newcommand{\ImportTok}[1]{{#1}}
    \newcommand{\DocumentationTok}[1]{\textcolor[rgb]{0.73,0.13,0.13}{\textit{{#1}}}}
    \newcommand{\AnnotationTok}[1]{\textcolor[rgb]{0.38,0.63,0.69}{\textbf{\textit{{#1}}}}}
    \newcommand{\CommentVarTok}[1]{\textcolor[rgb]{0.38,0.63,0.69}{\textbf{\textit{{#1}}}}}
    \newcommand{\VariableTok}[1]{\textcolor[rgb]{0.10,0.09,0.49}{{#1}}}
    \newcommand{\ControlFlowTok}[1]{\textcolor[rgb]{0.00,0.44,0.13}{\textbf{{#1}}}}
    \newcommand{\OperatorTok}[1]{\textcolor[rgb]{0.40,0.40,0.40}{{#1}}}
    \newcommand{\BuiltInTok}[1]{{#1}}
    \newcommand{\ExtensionTok}[1]{{#1}}
    \newcommand{\PreprocessorTok}[1]{\textcolor[rgb]{0.74,0.48,0.00}{{#1}}}
    \newcommand{\AttributeTok}[1]{\textcolor[rgb]{0.49,0.56,0.16}{{#1}}}
    \newcommand{\InformationTok}[1]{\textcolor[rgb]{0.38,0.63,0.69}{\textbf{\textit{{#1}}}}}
    \newcommand{\WarningTok}[1]{\textcolor[rgb]{0.38,0.63,0.69}{\textbf{\textit{{#1}}}}}
    
    
    % Define a nice break command that doesn't care if a line doesn't already
    % exist.
    \def\br{\hspace*{\fill} \\* }
    % Math Jax compatibility definitions
    \def\gt{>}
    \def\lt{<}
    \let\Oldtex\TeX
    \let\Oldlatex\LaTeX
    \renewcommand{\TeX}{\textrm{\Oldtex}}
    \renewcommand{\LaTeX}{\textrm{\Oldlatex}}
    % Document parameters
    % Document title
    \title{%
    Knapsack Problem using Brute Force \\ and Dynamic Programming \\
    \large CSE 401: Artificial Intelligence }

    \author{%
      Samyak Jain \\
      \small A2305217638 \\
      \small 7CSE 8Y}
    \date{September 23rd, 2020}
    
    
    
% Pygments definitions
\makeatletter
\def\PY@reset{\let\PY@it=\relax \let\PY@bf=\relax%
    \let\PY@ul=\relax \let\PY@tc=\relax%
    \let\PY@bc=\relax \let\PY@ff=\relax}
\def\PY@tok#1{\csname PY@tok@#1\endcsname}
\def\PY@toks#1+{\ifx\relax#1\empty\else%
    \PY@tok{#1}\expandafter\PY@toks\fi}
\def\PY@do#1{\PY@bc{\PY@tc{\PY@ul{%
    \PY@it{\PY@bf{\PY@ff{#1}}}}}}}
\def\PY#1#2{\PY@reset\PY@toks#1+\relax+\PY@do{#2}}

\expandafter\def\csname PY@tok@w\endcsname{\def\PY@tc##1{\textcolor[rgb]{0.73,0.73,0.73}{##1}}}
\expandafter\def\csname PY@tok@c\endcsname{\let\PY@it=\textit\def\PY@tc##1{\textcolor[rgb]{0.25,0.50,0.50}{##1}}}
\expandafter\def\csname PY@tok@cp\endcsname{\def\PY@tc##1{\textcolor[rgb]{0.74,0.48,0.00}{##1}}}
\expandafter\def\csname PY@tok@k\endcsname{\let\PY@bf=\textbf\def\PY@tc##1{\textcolor[rgb]{0.00,0.50,0.00}{##1}}}
\expandafter\def\csname PY@tok@kp\endcsname{\def\PY@tc##1{\textcolor[rgb]{0.00,0.50,0.00}{##1}}}
\expandafter\def\csname PY@tok@kt\endcsname{\def\PY@tc##1{\textcolor[rgb]{0.69,0.00,0.25}{##1}}}
\expandafter\def\csname PY@tok@o\endcsname{\def\PY@tc##1{\textcolor[rgb]{0.40,0.40,0.40}{##1}}}
\expandafter\def\csname PY@tok@ow\endcsname{\let\PY@bf=\textbf\def\PY@tc##1{\textcolor[rgb]{0.67,0.13,1.00}{##1}}}
\expandafter\def\csname PY@tok@nb\endcsname{\def\PY@tc##1{\textcolor[rgb]{0.00,0.50,0.00}{##1}}}
\expandafter\def\csname PY@tok@nf\endcsname{\def\PY@tc##1{\textcolor[rgb]{0.00,0.00,1.00}{##1}}}
\expandafter\def\csname PY@tok@nc\endcsname{\let\PY@bf=\textbf\def\PY@tc##1{\textcolor[rgb]{0.00,0.00,1.00}{##1}}}
\expandafter\def\csname PY@tok@nn\endcsname{\let\PY@bf=\textbf\def\PY@tc##1{\textcolor[rgb]{0.00,0.00,1.00}{##1}}}
\expandafter\def\csname PY@tok@ne\endcsname{\let\PY@bf=\textbf\def\PY@tc##1{\textcolor[rgb]{0.82,0.25,0.23}{##1}}}
\expandafter\def\csname PY@tok@nv\endcsname{\def\PY@tc##1{\textcolor[rgb]{0.10,0.09,0.49}{##1}}}
\expandafter\def\csname PY@tok@no\endcsname{\def\PY@tc##1{\textcolor[rgb]{0.53,0.00,0.00}{##1}}}
\expandafter\def\csname PY@tok@nl\endcsname{\def\PY@tc##1{\textcolor[rgb]{0.63,0.63,0.00}{##1}}}
\expandafter\def\csname PY@tok@ni\endcsname{\let\PY@bf=\textbf\def\PY@tc##1{\textcolor[rgb]{0.60,0.60,0.60}{##1}}}
\expandafter\def\csname PY@tok@na\endcsname{\def\PY@tc##1{\textcolor[rgb]{0.49,0.56,0.16}{##1}}}
\expandafter\def\csname PY@tok@nt\endcsname{\let\PY@bf=\textbf\def\PY@tc##1{\textcolor[rgb]{0.00,0.50,0.00}{##1}}}
\expandafter\def\csname PY@tok@nd\endcsname{\def\PY@tc##1{\textcolor[rgb]{0.67,0.13,1.00}{##1}}}
\expandafter\def\csname PY@tok@s\endcsname{\def\PY@tc##1{\textcolor[rgb]{0.73,0.13,0.13}{##1}}}
\expandafter\def\csname PY@tok@sd\endcsname{\let\PY@it=\textit\def\PY@tc##1{\textcolor[rgb]{0.73,0.13,0.13}{##1}}}
\expandafter\def\csname PY@tok@si\endcsname{\let\PY@bf=\textbf\def\PY@tc##1{\textcolor[rgb]{0.73,0.40,0.53}{##1}}}
\expandafter\def\csname PY@tok@se\endcsname{\let\PY@bf=\textbf\def\PY@tc##1{\textcolor[rgb]{0.73,0.40,0.13}{##1}}}
\expandafter\def\csname PY@tok@sr\endcsname{\def\PY@tc##1{\textcolor[rgb]{0.73,0.40,0.53}{##1}}}
\expandafter\def\csname PY@tok@ss\endcsname{\def\PY@tc##1{\textcolor[rgb]{0.10,0.09,0.49}{##1}}}
\expandafter\def\csname PY@tok@sx\endcsname{\def\PY@tc##1{\textcolor[rgb]{0.00,0.50,0.00}{##1}}}
\expandafter\def\csname PY@tok@m\endcsname{\def\PY@tc##1{\textcolor[rgb]{0.40,0.40,0.40}{##1}}}
\expandafter\def\csname PY@tok@gh\endcsname{\let\PY@bf=\textbf\def\PY@tc##1{\textcolor[rgb]{0.00,0.00,0.50}{##1}}}
\expandafter\def\csname PY@tok@gu\endcsname{\let\PY@bf=\textbf\def\PY@tc##1{\textcolor[rgb]{0.50,0.00,0.50}{##1}}}
\expandafter\def\csname PY@tok@gd\endcsname{\def\PY@tc##1{\textcolor[rgb]{0.63,0.00,0.00}{##1}}}
\expandafter\def\csname PY@tok@gi\endcsname{\def\PY@tc##1{\textcolor[rgb]{0.00,0.63,0.00}{##1}}}
\expandafter\def\csname PY@tok@gr\endcsname{\def\PY@tc##1{\textcolor[rgb]{1.00,0.00,0.00}{##1}}}
\expandafter\def\csname PY@tok@ge\endcsname{\let\PY@it=\textit}
\expandafter\def\csname PY@tok@gs\endcsname{\let\PY@bf=\textbf}
\expandafter\def\csname PY@tok@gp\endcsname{\let\PY@bf=\textbf\def\PY@tc##1{\textcolor[rgb]{0.00,0.00,0.50}{##1}}}
\expandafter\def\csname PY@tok@go\endcsname{\def\PY@tc##1{\textcolor[rgb]{0.53,0.53,0.53}{##1}}}
\expandafter\def\csname PY@tok@gt\endcsname{\def\PY@tc##1{\textcolor[rgb]{0.00,0.27,0.87}{##1}}}
\expandafter\def\csname PY@tok@err\endcsname{\def\PY@bc##1{\setlength{\fboxsep}{0pt}\fcolorbox[rgb]{1.00,0.00,0.00}{1,1,1}{\strut ##1}}}
\expandafter\def\csname PY@tok@kc\endcsname{\let\PY@bf=\textbf\def\PY@tc##1{\textcolor[rgb]{0.00,0.50,0.00}{##1}}}
\expandafter\def\csname PY@tok@kd\endcsname{\let\PY@bf=\textbf\def\PY@tc##1{\textcolor[rgb]{0.00,0.50,0.00}{##1}}}
\expandafter\def\csname PY@tok@kn\endcsname{\let\PY@bf=\textbf\def\PY@tc##1{\textcolor[rgb]{0.00,0.50,0.00}{##1}}}
\expandafter\def\csname PY@tok@kr\endcsname{\let\PY@bf=\textbf\def\PY@tc##1{\textcolor[rgb]{0.00,0.50,0.00}{##1}}}
\expandafter\def\csname PY@tok@bp\endcsname{\def\PY@tc##1{\textcolor[rgb]{0.00,0.50,0.00}{##1}}}
\expandafter\def\csname PY@tok@fm\endcsname{\def\PY@tc##1{\textcolor[rgb]{0.00,0.00,1.00}{##1}}}
\expandafter\def\csname PY@tok@vc\endcsname{\def\PY@tc##1{\textcolor[rgb]{0.10,0.09,0.49}{##1}}}
\expandafter\def\csname PY@tok@vg\endcsname{\def\PY@tc##1{\textcolor[rgb]{0.10,0.09,0.49}{##1}}}
\expandafter\def\csname PY@tok@vi\endcsname{\def\PY@tc##1{\textcolor[rgb]{0.10,0.09,0.49}{##1}}}
\expandafter\def\csname PY@tok@vm\endcsname{\def\PY@tc##1{\textcolor[rgb]{0.10,0.09,0.49}{##1}}}
\expandafter\def\csname PY@tok@sa\endcsname{\def\PY@tc##1{\textcolor[rgb]{0.73,0.13,0.13}{##1}}}
\expandafter\def\csname PY@tok@sb\endcsname{\def\PY@tc##1{\textcolor[rgb]{0.73,0.13,0.13}{##1}}}
\expandafter\def\csname PY@tok@sc\endcsname{\def\PY@tc##1{\textcolor[rgb]{0.73,0.13,0.13}{##1}}}
\expandafter\def\csname PY@tok@dl\endcsname{\def\PY@tc##1{\textcolor[rgb]{0.73,0.13,0.13}{##1}}}
\expandafter\def\csname PY@tok@s2\endcsname{\def\PY@tc##1{\textcolor[rgb]{0.73,0.13,0.13}{##1}}}
\expandafter\def\csname PY@tok@sh\endcsname{\def\PY@tc##1{\textcolor[rgb]{0.73,0.13,0.13}{##1}}}
\expandafter\def\csname PY@tok@s1\endcsname{\def\PY@tc##1{\textcolor[rgb]{0.73,0.13,0.13}{##1}}}
\expandafter\def\csname PY@tok@mb\endcsname{\def\PY@tc##1{\textcolor[rgb]{0.40,0.40,0.40}{##1}}}
\expandafter\def\csname PY@tok@mf\endcsname{\def\PY@tc##1{\textcolor[rgb]{0.40,0.40,0.40}{##1}}}
\expandafter\def\csname PY@tok@mh\endcsname{\def\PY@tc##1{\textcolor[rgb]{0.40,0.40,0.40}{##1}}}
\expandafter\def\csname PY@tok@mi\endcsname{\def\PY@tc##1{\textcolor[rgb]{0.40,0.40,0.40}{##1}}}
\expandafter\def\csname PY@tok@il\endcsname{\def\PY@tc##1{\textcolor[rgb]{0.40,0.40,0.40}{##1}}}
\expandafter\def\csname PY@tok@mo\endcsname{\def\PY@tc##1{\textcolor[rgb]{0.40,0.40,0.40}{##1}}}
\expandafter\def\csname PY@tok@ch\endcsname{\let\PY@it=\textit\def\PY@tc##1{\textcolor[rgb]{0.25,0.50,0.50}{##1}}}
\expandafter\def\csname PY@tok@cm\endcsname{\let\PY@it=\textit\def\PY@tc##1{\textcolor[rgb]{0.25,0.50,0.50}{##1}}}
\expandafter\def\csname PY@tok@cpf\endcsname{\let\PY@it=\textit\def\PY@tc##1{\textcolor[rgb]{0.25,0.50,0.50}{##1}}}
\expandafter\def\csname PY@tok@c1\endcsname{\let\PY@it=\textit\def\PY@tc##1{\textcolor[rgb]{0.25,0.50,0.50}{##1}}}
\expandafter\def\csname PY@tok@cs\endcsname{\let\PY@it=\textit\def\PY@tc##1{\textcolor[rgb]{0.25,0.50,0.50}{##1}}}

\def\PYZbs{\char`\\}
\def\PYZus{\char`\_}
\def\PYZob{\char`\{}
\def\PYZcb{\char`\}}
\def\PYZca{\char`\^}
\def\PYZam{\char`\&}
\def\PYZlt{\char`\<}
\def\PYZgt{\char`\>}
\def\PYZsh{\char`\#}
\def\PYZpc{\char`\%}
\def\PYZdl{\char`\$}
\def\PYZhy{\char`\-}
\def\PYZsq{\char`\'}
\def\PYZdq{\char`\"}
\def\PYZti{\char`\~}
% for compatibility with earlier versions
\def\PYZat{@}
\def\PYZlb{[}
\def\PYZrb{]}
\makeatother


    % For linebreaks inside Verbatim environment from package fancyvrb. 
    \makeatletter
        \newbox\Wrappedcontinuationbox 
        \newbox\Wrappedvisiblespacebox 
        \newcommand*\Wrappedvisiblespace {\textcolor{red}{\textvisiblespace}} 
        \newcommand*\Wrappedcontinuationsymbol {\textcolor{red}{\llap{\tiny$\m@th\hookrightarrow$}}} 
        \newcommand*\Wrappedcontinuationindent {3ex } 
        \newcommand*\Wrappedafterbreak {\kern\Wrappedcontinuationindent\copy\Wrappedcontinuationbox} 
        % Take advantage of the already applied Pygments mark-up to insert 
        % potential linebreaks for TeX processing. 
        %        {, <, #, %, $, ' and ": go to next line. 
        %        _, }, ^, &, >, - and ~: stay at end of broken line. 
        % Use of \textquotesingle for straight quote. 
        \newcommand*\Wrappedbreaksatspecials {% 
            \def\PYGZus{\discretionary{\char`\_}{\Wrappedafterbreak}{\char`\_}}% 
            \def\PYGZob{\discretionary{}{\Wrappedafterbreak\char`\{}{\char`\{}}% 
            \def\PYGZcb{\discretionary{\char`\}}{\Wrappedafterbreak}{\char`\}}}% 
            \def\PYGZca{\discretionary{\char`\^}{\Wrappedafterbreak}{\char`\^}}% 
            \def\PYGZam{\discretionary{\char`\&}{\Wrappedafterbreak}{\char`\&}}% 
            \def\PYGZlt{\discretionary{}{\Wrappedafterbreak\char`\<}{\char`\<}}% 
            \def\PYGZgt{\discretionary{\char`\>}{\Wrappedafterbreak}{\char`\>}}% 
            \def\PYGZsh{\discretionary{}{\Wrappedafterbreak\char`\#}{\char`\#}}% 
            \def\PYGZpc{\discretionary{}{\Wrappedafterbreak\char`\%}{\char`\%}}% 
            \def\PYGZdl{\discretionary{}{\Wrappedafterbreak\char`\$}{\char`\$}}% 
            \def\PYGZhy{\discretionary{\char`\-}{\Wrappedafterbreak}{\char`\-}}% 
            \def\PYGZsq{\discretionary{}{\Wrappedafterbreak\textquotesingle}{\textquotesingle}}% 
            \def\PYGZdq{\discretionary{}{\Wrappedafterbreak\char`\"}{\char`\"}}% 
            \def\PYGZti{\discretionary{\char`\~}{\Wrappedafterbreak}{\char`\~}}% 
        } 
        % Some characters . , ; ? ! / are not pygmentized. 
        % This macro makes them "active" and they will insert potential linebreaks 
        \newcommand*\Wrappedbreaksatpunct {% 
            \lccode`\~`\.\lowercase{\def~}{\discretionary{\hbox{\char`\.}}{\Wrappedafterbreak}{\hbox{\char`\.}}}% 
            \lccode`\~`\,\lowercase{\def~}{\discretionary{\hbox{\char`\,}}{\Wrappedafterbreak}{\hbox{\char`\,}}}% 
            \lccode`\~`\;\lowercase{\def~}{\discretionary{\hbox{\char`\;}}{\Wrappedafterbreak}{\hbox{\char`\;}}}% 
            \lccode`\~`\:\lowercase{\def~}{\discretionary{\hbox{\char`\:}}{\Wrappedafterbreak}{\hbox{\char`\:}}}% 
            \lccode`\~`\?\lowercase{\def~}{\discretionary{\hbox{\char`\?}}{\Wrappedafterbreak}{\hbox{\char`\?}}}% 
            \lccode`\~`\!\lowercase{\def~}{\discretionary{\hbox{\char`\!}}{\Wrappedafterbreak}{\hbox{\char`\!}}}% 
            \lccode`\~`\/\lowercase{\def~}{\discretionary{\hbox{\char`\/}}{\Wrappedafterbreak}{\hbox{\char`\/}}}% 
            \catcode`\.\active
            \catcode`\,\active 
            \catcode`\;\active
            \catcode`\:\active
            \catcode`\?\active
            \catcode`\!\active
            \catcode`\/\active 
            \lccode`\~`\~ 	
        }
    \makeatother

    \let\OriginalVerbatim=\Verbatim
    \makeatletter
    \renewcommand{\Verbatim}[1][1]{%
        %\parskip\z@skip
        \sbox\Wrappedcontinuationbox {\Wrappedcontinuationsymbol}%
        \sbox\Wrappedvisiblespacebox {\FV@SetupFont\Wrappedvisiblespace}%
        \def\FancyVerbFormatLine ##1{\hsize\linewidth
            \vtop{\raggedright\hyphenpenalty\z@\exhyphenpenalty\z@
                \doublehyphendemerits\z@\finalhyphendemerits\z@
                \strut ##1\strut}%
        }%
        % If the linebreak is at a space, the latter will be displayed as visible
        % space at end of first line, and a continuation symbol starts next line.
        % Stretch/shrink are however usually zero for typewriter font.
        \def\FV@Space {%
            \nobreak\hskip\z@ plus\fontdimen3\font minus\fontdimen4\font
            \discretionary{\copy\Wrappedvisiblespacebox}{\Wrappedafterbreak}
            {\kern\fontdimen2\font}%
        }%
        
        % Allow breaks at special characters using \PYG... macros.
        \Wrappedbreaksatspecials
        % Breaks at punctuation characters . , ; ? ! and / need catcode=\active 	
        \OriginalVerbatim[#1,codes*=\Wrappedbreaksatpunct]%
    }
    \makeatother

    % Exact colors from NB
    \definecolor{incolor}{HTML}{303F9F}
    \definecolor{outcolor}{HTML}{D84315}
    \definecolor{cellborder}{HTML}{CFCFCF}
    \definecolor{cellbackground}{HTML}{F7F7F7}
    
    % prompt
    \makeatletter
    \newcommand{\boxspacing}{\kern\kvtcb@left@rule\kern\kvtcb@boxsep}
    \makeatother
    \newcommand{\prompt}[4]{
        \ttfamily\llap{{\color{#2}[#3]:\hspace{3pt}#4}}\vspace{-\baselineskip}
    }
    

    
    % Prevent overflowing lines due to hard-to-break entities
    \sloppy 
    % Setup hyperref package
    \hypersetup{
      breaklinks=true,  % so long urls are correctly broken across lines
      colorlinks=true,
      urlcolor=urlcolor,
      linkcolor=linkcolor,
      citecolor=citecolor,
      }
    % Slightly bigger margins than the latex defaults
    
    \geometry{verbose,tmargin=1in,bmargin=1in,lmargin=1in,rmargin=1in}
    
    

\begin{document}
    

    \maketitle

\hypertarget{knapsack-problem}{%
\section{Knapsack Problem}\label{knapsack-problem}}

\textbf{Implement a Knapsack problem using Brute Force Method and Dynamic
Programming}

    The knapsack problem is an optimization problem that takes a common
computational need---finding the best use of limited resources given a
finite set of usage options---and spins it into a fun story. A thief
enters a shop with the intent to steal. He has a knapsack, and he is
limited in what he can steal by the capacity of the knapsack. How does
he figure out what to put into the knapsack?


\subsection{Brute Force Approach}

    If we tried to solve this problem using a brute-force approach, we would
look at every combination of items available to be put in the knapsack.
For the mathematically inclined, this is known as a powerset, and a
powerset of a set (in our case, the set of items) has \(2^{N}\)
different possible subsets, where \(N\) is the number of items.
Therefore, we would need to analyze \(2^{N}\) combinations
\((O(2^{N}))\). This is okay for a small number of items, but it is
untenable for a large number. Any approach that solves a problem using
an exponential number of steps is an approach we want to avoid.

    \begin{tcolorbox}[breakable, size=fbox, boxrule=1pt, pad at break*=1mm,colback=cellbackground, colframe=cellborder]
\prompt{In}{incolor}{1}{\boxspacing}
\begin{Verbatim}[commandchars=\\\{\}]
\PY{k+kn}{from} \PY{n+nn}{itertools} \PY{k}{import} \PY{n}{product}
\PY{k+kn}{from} \PY{n+nn}{collections} \PY{k}{import} \PY{n}{namedtuple}
\PY{k}{try}\PY{p}{:}
    \PY{k+kn}{from} \PY{n+nn}{itertools} \PY{k}{import} \PY{n}{izip}
\PY{k}{except} \PY{n+ne}{ImportError}\PY{p}{:}
    \PY{n}{izip} \PY{o}{=} \PY{n+nb}{zip}
\end{Verbatim}
\end{tcolorbox}

    \begin{tcolorbox}[breakable, size=fbox, boxrule=1pt, pad at break*=1mm,colback=cellbackground, colframe=cellborder]
\prompt{In}{incolor}{2}{\boxspacing}
\begin{Verbatim}[commandchars=\\\{\}]
\PY{n}{Reward} \PY{o}{=} \PY{n}{namedtuple}\PY{p}{(}\PY{l+s+s1}{\PYZsq{}}\PY{l+s+s1}{Reward}\PY{l+s+s1}{\PYZsq{}}\PY{p}{,} \PY{l+s+s1}{\PYZsq{}}\PY{l+s+s1}{name value weight volume}\PY{l+s+s1}{\PYZsq{}}\PY{p}{)}

\PY{n}{bagpack} \PY{o}{=}   \PY{n}{Reward}\PY{p}{(}\PY{l+s+s1}{\PYZsq{}}\PY{l+s+s1}{bagpack}\PY{l+s+s1}{\PYZsq{}}\PY{p}{,}  \PY{l+m+mi}{0}\PY{p}{,} \PY{l+m+mf}{25.0}\PY{p}{,} \PY{l+m+mf}{0.25}\PY{p}{)}

\PY{n}{items} \PY{o}{=} \PY{p}{[}\PY{n}{Reward}\PY{p}{(}\PY{l+s+s1}{\PYZsq{}}\PY{l+s+s1}{laptop}\PY{l+s+s1}{\PYZsq{}}\PY{p}{,}    \PY{l+m+mi}{3000}\PY{p}{,}  \PY{l+m+mf}{0.3}\PY{p}{,} \PY{l+m+mf}{0.025}\PY{p}{)}\PY{p}{,}
         \PY{n}{Reward}\PY{p}{(}\PY{l+s+s1}{\PYZsq{}}\PY{l+s+s1}{printer}\PY{l+s+s1}{\PYZsq{}}\PY{p}{,}   \PY{l+m+mi}{1800}\PY{p}{,}  \PY{l+m+mf}{0.2}\PY{p}{,} \PY{l+m+mf}{0.015}\PY{p}{)}\PY{p}{,}
         \PY{n}{Reward}\PY{p}{(}\PY{l+s+s1}{\PYZsq{}}\PY{l+s+s1}{headphone}\PY{l+s+s1}{\PYZsq{}}\PY{p}{,} \PY{l+m+mi}{2500}\PY{p}{,}  \PY{l+m+mf}{2.0}\PY{p}{,} \PY{l+m+mf}{0.002}\PY{p}{)}\PY{p}{]}
\end{Verbatim}
\end{tcolorbox}

    \begin{tcolorbox}[breakable, size=fbox, boxrule=1pt, pad at break*=1mm,colback=cellbackground, colframe=cellborder]
\prompt{In}{incolor}{3}{\boxspacing}
\begin{Verbatim}[commandchars=\\\{\}]
\PY{k}{def} \PY{n+nf}{tot\PYZus{}value}\PY{p}{(}\PY{n}{items\PYZus{}count}\PY{p}{)}\PY{p}{:}
    \PY{l+s+sd}{\PYZdq{}\PYZdq{}\PYZdq{}}
\PY{l+s+sd}{    Given the count of each item in the sack return \PYZhy{}1 if they can\PYZsq{}t be carried or their total value.}
\PY{l+s+sd}{ }
\PY{l+s+sd}{    (also return the negative of the weight and the volume so taking the max of a series of return}
\PY{l+s+sd}{    values will minimise the weight if values tie, and minimise the volume if values and weights tie).}
\PY{l+s+sd}{    \PYZdq{}\PYZdq{}\PYZdq{}}
    \PY{k}{global} \PY{n}{items}\PY{p}{,} \PY{n}{bagpack}
    \PY{n}{weight} \PY{o}{=} \PY{n+nb}{sum}\PY{p}{(}\PY{n}{n} \PY{o}{*} \PY{n}{item}\PY{o}{.}\PY{n}{weight} \PY{k}{for} \PY{n}{n}\PY{p}{,} \PY{n}{item} \PY{o+ow}{in} \PY{n}{izip}\PY{p}{(}\PY{n}{items\PYZus{}count}\PY{p}{,} \PY{n}{items}\PY{p}{)}\PY{p}{)}
    \PY{n}{volume} \PY{o}{=} \PY{n+nb}{sum}\PY{p}{(}\PY{n}{n} \PY{o}{*} \PY{n}{item}\PY{o}{.}\PY{n}{volume} \PY{k}{for} \PY{n}{n}\PY{p}{,} \PY{n}{item} \PY{o+ow}{in} \PY{n}{izip}\PY{p}{(}\PY{n}{items\PYZus{}count}\PY{p}{,} \PY{n}{items}\PY{p}{)}\PY{p}{)}
    \PY{k}{if} \PY{n}{weight} \PY{o}{\PYZlt{}}\PY{o}{=} \PY{n}{bagpack}\PY{o}{.}\PY{n}{weight} \PY{o+ow}{and} \PY{n}{volume} \PY{o}{\PYZlt{}}\PY{o}{=} \PY{n}{bagpack}\PY{o}{.}\PY{n}{volume}\PY{p}{:}
        \PY{k}{return} \PY{n+nb}{sum}\PY{p}{(}\PY{n}{n} \PY{o}{*} \PY{n}{item}\PY{o}{.}\PY{n}{value} \PY{k}{for} \PY{n}{n}\PY{p}{,} \PY{n}{item} \PY{o+ow}{in} \PY{n}{izip}\PY{p}{(}\PY{n}{items\PYZus{}count}\PY{p}{,} \PY{n}{items}\PY{p}{)}\PY{p}{)}\PY{p}{,} \PY{o}{\PYZhy{}}\PY{n}{weight}\PY{p}{,} \PY{o}{\PYZhy{}}\PY{n}{volume}    
    \PY{k}{else}\PY{p}{:}
        \PY{k}{return} \PY{o}{\PYZhy{}}\PY{l+m+mi}{1}\PY{p}{,} \PY{l+m+mi}{0}\PY{p}{,} \PY{l+m+mi}{0}
\end{Verbatim}
\end{tcolorbox}

    \begin{tcolorbox}[breakable, size=fbox, boxrule=1pt, pad at break*=1mm,colback=cellbackground, colframe=cellborder]
\prompt{In}{incolor}{4}{\boxspacing}
\begin{Verbatim}[commandchars=\\\{\}]
\PY{k}{def} \PY{n+nf}{knapsack}\PY{p}{(}\PY{p}{)}\PY{p}{:}
    \PY{k}{global} \PY{n}{items}\PY{p}{,} \PY{n}{bagpack} 
    \PY{c+c1}{\PYZsh{} find max of any one item}
    \PY{n}{max1} \PY{o}{=} \PY{p}{[}\PY{n+nb}{min}\PY{p}{(}\PY{n+nb}{int}\PY{p}{(}\PY{n}{bagpack}\PY{o}{.}\PY{n}{weight} \PY{o}{/}\PY{o}{/} \PY{n}{item}\PY{o}{.}\PY{n}{weight}\PY{p}{)}\PY{p}{,} \PY{n+nb}{int}\PY{p}{(}\PY{n}{bagpack}\PY{o}{.}\PY{n}{volume} \PY{o}{/}\PY{o}{/} \PY{n}{item}\PY{o}{.}\PY{n}{volume}\PY{p}{)}\PY{p}{)} \PY{k}{for} \PY{n}{item} \PY{o+ow}{in} \PY{n}{items}\PY{p}{]}
 
    \PY{c+c1}{\PYZsh{} Try all combinations of reward items from 0 up to max1}
    \PY{k}{return} \PY{n+nb}{max}\PY{p}{(}\PY{n}{product}\PY{p}{(}\PY{o}{*}\PY{p}{[}\PY{n+nb}{range}\PY{p}{(}\PY{n}{n} \PY{o}{+} \PY{l+m+mi}{1}\PY{p}{)} \PY{k}{for} \PY{n}{n} \PY{o+ow}{in} \PY{n}{max1}\PY{p}{]}\PY{p}{)}\PY{p}{,} \PY{n}{key}\PY{o}{=}\PY{n}{tot\PYZus{}value}\PY{p}{)}
\end{Verbatim}
\end{tcolorbox}

    \begin{tcolorbox}[breakable, size=fbox, boxrule=1pt, pad at break*=1mm,colback=cellbackground, colframe=cellborder]
\prompt{In}{incolor}{5}{\boxspacing}
\begin{Verbatim}[commandchars=\\\{\}]
\PY{k+kn}{import} \PY{n+nn}{time}

\PY{n}{start} \PY{o}{=} \PY{n}{time}\PY{o}{.}\PY{n}{time}\PY{p}{(}\PY{p}{)}

\PY{n}{max\PYZus{}items} \PY{o}{=} \PY{n}{knapsack}\PY{p}{(}\PY{p}{)}
\PY{n}{maxvalue}\PY{p}{,} \PY{n}{max\PYZus{}weight}\PY{p}{,} \PY{n}{max\PYZus{}volume} \PY{o}{=} \PY{n}{tot\PYZus{}value}\PY{p}{(}\PY{n}{max\PYZus{}items}\PY{p}{)}
\PY{n}{max\PYZus{}weight} \PY{o}{=} \PY{o}{\PYZhy{}}\PY{n}{max\PYZus{}weight}
\PY{n}{max\PYZus{}volume} \PY{o}{=} \PY{o}{\PYZhy{}}\PY{n}{max\PYZus{}volume}
 
\PY{n+nb}{print}\PY{p}{(}\PY{l+s+s2}{\PYZdq{}}\PY{l+s+s2}{The maximum value achievable (by exhaustive search) is }\PY{l+s+si}{\PYZpc{}g}\PY{l+s+s2}{.}\PY{l+s+s2}{\PYZdq{}} \PY{o}{\PYZpc{}} \PY{n}{maxvalue}\PY{p}{)}
\PY{n}{item\PYZus{}names} \PY{o}{=} \PY{l+s+s2}{\PYZdq{}}\PY{l+s+s2}{, }\PY{l+s+s2}{\PYZdq{}}\PY{o}{.}\PY{n}{join}\PY{p}{(}\PY{n}{item}\PY{o}{.}\PY{n}{name} \PY{k}{for} \PY{n}{item} \PY{o+ow}{in} \PY{n}{items}\PY{p}{)}
\PY{n+nb}{print}\PY{p}{(}\PY{l+s+s2}{\PYZdq{}}\PY{l+s+s2}{  The number of }\PY{l+s+si}{\PYZpc{}s}\PY{l+s+s2}{ items to achieve this is: }\PY{l+s+si}{\PYZpc{}s}\PY{l+s+s2}{, respectively.}\PY{l+s+s2}{\PYZdq{}} \PY{o}{\PYZpc{}} \PY{p}{(}\PY{n}{item\PYZus{}names}\PY{p}{,} \PY{n}{max\PYZus{}items}\PY{p}{)}\PY{p}{)}
\PY{n+nb}{print}\PY{p}{(}\PY{l+s+s2}{\PYZdq{}}\PY{l+s+s2}{  The weight to carry is }\PY{l+s+si}{\PYZpc{}.3g}\PY{l+s+s2}{, and the volume used is }\PY{l+s+si}{\PYZpc{}.3g}\PY{l+s+s2}{.}\PY{l+s+s2}{\PYZdq{}} \PY{o}{\PYZpc{}} \PY{p}{(}\PY{n}{max\PYZus{}weight}\PY{p}{,} \PY{n}{max\PYZus{}volume}\PY{p}{)}\PY{p}{)}

\PY{n}{end} \PY{o}{=} \PY{n}{time}\PY{o}{.}\PY{n}{time}\PY{p}{(}\PY{p}{)}
\PY{n+nb}{print}\PY{p}{(}\PY{n}{f}\PY{l+s+s2}{\PYZdq{}}\PY{l+s+se}{\PYZbs{}n}\PY{l+s+s2}{The total execution time taken is }\PY{l+s+s2}{\PYZob{}}\PY{l+s+s2}{end\PYZhy{}start\PYZcb{}.}\PY{l+s+s2}{\PYZdq{}}\PY{p}{)}
\end{Verbatim}
\end{tcolorbox}

    \begin{Verbatim}[commandchars=\\\{\}]
The maximum value achievable (by exhaustive search) is 54500.
  The number of laptop, printer, headphone items to achieve this is: (9, 0, 11),
respectively.
  The weight to carry is 24.7, and the volume used is 0.247.

The total execution time taken is 0.0247344970703125.
    \end{Verbatim}

    ~\\
    \noindent\rule{16.5cm}{0.4pt}

~\\
\subsection{Dynamic Programming Approach}

\textbf{Implement a Knapsack problem using Dynamic Programming. Compare the
execution time of brute-force and dynamic programming algorithms.}

Instead, use a technique known as dynamic programming, which is similar
in concept to memoization. Instead of solving a problem outright with a
brute-force approach, in dynamic programming one solves subproblems that
make up the larger problem, stores those results, and utilizes those
stored results to solve the larger problem. As long as the capacity of
the knapsack is considered in discrete steps, the problem can be solved
with dynamic programming.

    \begin{tcolorbox}[breakable, size=fbox, boxrule=1pt, pad at break*=1mm,colback=cellbackground, colframe=cellborder]
\prompt{In}{incolor}{6}{\boxspacing}
\begin{Verbatim}[commandchars=\\\{\}]
\PY{k+kn}{from} \PY{n+nn}{itertools} \PY{k}{import} \PY{n}{product}
\PY{k+kn}{from} \PY{n+nn}{collections} \PY{k}{import} \PY{n}{namedtuple}
\PY{k}{try}\PY{p}{:}
    \PY{k+kn}{from} \PY{n+nn}{itertools} \PY{k}{import} \PY{n}{izip}
\PY{k}{except} \PY{n+ne}{ImportError}\PY{p}{:}
    \PY{n}{izip} \PY{o}{=} \PY{n+nb}{zip}
\end{Verbatim}
\end{tcolorbox}

    \begin{tcolorbox}[breakable, size=fbox, boxrule=1pt, pad at break*=1mm,colback=cellbackground, colframe=cellborder]
\prompt{In}{incolor}{7}{\boxspacing}
\begin{Verbatim}[commandchars=\\\{\}]
\PY{n}{Reward} \PY{o}{=} \PY{n}{namedtuple}\PY{p}{(}\PY{l+s+s1}{\PYZsq{}}\PY{l+s+s1}{Reward}\PY{l+s+s1}{\PYZsq{}}\PY{p}{,} \PY{l+s+s1}{\PYZsq{}}\PY{l+s+s1}{name value weight volume}\PY{l+s+s1}{\PYZsq{}}\PY{p}{)}

\PY{n}{bagpack} \PY{o}{=}   \PY{n}{Reward}\PY{p}{(}\PY{l+s+s1}{\PYZsq{}}\PY{l+s+s1}{bagpack}\PY{l+s+s1}{\PYZsq{}}\PY{p}{,}  \PY{l+m+mi}{0}\PY{p}{,} \PY{l+m+mi}{250}\PY{p}{,} \PY{l+m+mi}{250}\PY{p}{)}

\PY{n}{items} \PY{o}{=} \PY{p}{[}\PY{n}{Reward}\PY{p}{(}\PY{l+s+s1}{\PYZsq{}}\PY{l+s+s1}{laptop}\PY{l+s+s1}{\PYZsq{}}\PY{p}{,}    \PY{l+m+mi}{3000}\PY{p}{,}  \PY{l+m+mi}{3}\PY{p}{,} \PY{l+m+mi}{25}\PY{p}{)}\PY{p}{,}
         \PY{n}{Reward}\PY{p}{(}\PY{l+s+s1}{\PYZsq{}}\PY{l+s+s1}{printer}\PY{l+s+s1}{\PYZsq{}}\PY{p}{,}   \PY{l+m+mi}{1800}\PY{p}{,}  \PY{l+m+mi}{2}\PY{p}{,} \PY{l+m+mi}{15}\PY{p}{)}\PY{p}{,}
         \PY{n}{Reward}\PY{p}{(}\PY{l+s+s1}{\PYZsq{}}\PY{l+s+s1}{headphone}\PY{l+s+s1}{\PYZsq{}}\PY{p}{,} \PY{l+m+mi}{2500}\PY{p}{,}  \PY{l+m+mi}{20}\PY{p}{,} \PY{l+m+mi}{2}\PY{p}{)}\PY{p}{]}
\end{Verbatim}
\end{tcolorbox}

    \begin{tcolorbox}[breakable, size=fbox, boxrule=1pt, pad at break*=1mm,colback=cellbackground, colframe=cellborder]
\prompt{In}{incolor}{8}{\boxspacing}
\begin{Verbatim}[commandchars=\\\{\}]
\PY{k}{def} \PY{n+nf}{tot\PYZus{}value}\PY{p}{(}\PY{n}{items\PYZus{}count}\PY{p}{,} \PY{n}{items}\PY{p}{,} \PY{n}{bagpack}\PY{p}{)}\PY{p}{:}
    \PY{l+s+sd}{\PYZdq{}\PYZdq{}\PYZdq{}}
\PY{l+s+sd}{    Given the count of each item in the bagpack return \PYZhy{}1 if they can\PYZsq{}t be carried or their total value.}

\PY{l+s+sd}{    (also return the negative of the weight and the volume so taking the max of a series of return}
\PY{l+s+sd}{    values will minimise the weight if values tie, and minimise the volume if values and weights tie).}
\PY{l+s+sd}{    \PYZdq{}\PYZdq{}\PYZdq{}}
    \PY{n}{weight} \PY{o}{=} \PY{n+nb}{sum}\PY{p}{(}\PY{n}{n} \PY{o}{*} \PY{n}{item}\PY{o}{.}\PY{n}{weight} \PY{k}{for} \PY{n}{n}\PY{p}{,} \PY{n}{item} \PY{o+ow}{in} \PY{n}{izip}\PY{p}{(}\PY{n}{items\PYZus{}count}\PY{p}{,} \PY{n}{items}\PY{p}{)}\PY{p}{)}
    \PY{n}{volume} \PY{o}{=} \PY{n+nb}{sum}\PY{p}{(}\PY{n}{n} \PY{o}{*} \PY{n}{item}\PY{o}{.}\PY{n}{volume} \PY{k}{for} \PY{n}{n}\PY{p}{,} \PY{n}{item} \PY{o+ow}{in} \PY{n}{izip}\PY{p}{(}\PY{n}{items\PYZus{}count}\PY{p}{,} \PY{n}{items}\PY{p}{)}\PY{p}{)}
    \PY{k}{if} \PY{n}{weight} \PY{o}{\PYZlt{}}\PY{o}{=} \PY{n}{bagpack}\PY{o}{.}\PY{n}{weight} \PY{o+ow}{and} \PY{n}{volume} \PY{o}{\PYZlt{}}\PY{o}{=} \PY{n}{bagpack}\PY{o}{.}\PY{n}{volume}\PY{p}{:}
        \PY{k}{return} \PY{n+nb}{sum}\PY{p}{(}\PY{n}{n} \PY{o}{*} \PY{n}{item}\PY{o}{.}\PY{n}{value} \PY{k}{for} \PY{n}{n}\PY{p}{,} \PY{n}{item} \PY{o+ow}{in} \PY{n}{izip}\PY{p}{(}\PY{n}{items\PYZus{}count}\PY{p}{,} \PY{n}{items}\PY{p}{)}\PY{p}{)}\PY{p}{,} \PY{o}{\PYZhy{}}\PY{n}{weight}\PY{p}{,} \PY{o}{\PYZhy{}}\PY{n}{volume}
    \PY{k}{else}\PY{p}{:}
        \PY{k}{return} \PY{o}{\PYZhy{}}\PY{l+m+mi}{1}\PY{p}{,} \PY{l+m+mi}{0}\PY{p}{,} \PY{l+m+mi}{0}
\end{Verbatim}
\end{tcolorbox}

    \begin{tcolorbox}[breakable, size=fbox, boxrule=1pt, pad at break*=1mm,colback=cellbackground, colframe=cellborder]
\prompt{In}{incolor}{9}{\boxspacing}
\begin{Verbatim}[commandchars=\\\{\}]
\PY{k}{def} \PY{n+nf}{knapsack}\PY{p}{(}\PY{n}{items}\PY{p}{,} \PY{n}{bagpack}\PY{p}{)}\PY{p}{:}
    \PY{n}{table} \PY{o}{=} \PY{p}{[}\PY{p}{[}\PY{l+m+mi}{0}\PY{p}{]} \PY{o}{*} \PY{p}{(}\PY{n}{bagpack}\PY{o}{.}\PY{n}{volume} \PY{o}{+} \PY{l+m+mi}{1}\PY{p}{)} \PY{k}{for} \PY{n}{i} \PY{o+ow}{in} \PY{n+nb}{range}\PY{p}{(}\PY{n}{bagpack}\PY{o}{.}\PY{n}{weight} \PY{o}{+} \PY{l+m+mi}{1}\PY{p}{)}\PY{p}{]}

    \PY{k}{for} \PY{n}{w} \PY{o+ow}{in} \PY{n+nb}{range}\PY{p}{(}\PY{n}{bagpack}\PY{o}{.}\PY{n}{weight} \PY{o}{+} \PY{l+m+mi}{1}\PY{p}{)}\PY{p}{:}
        \PY{k}{for} \PY{n}{v} \PY{o+ow}{in} \PY{n+nb}{range}\PY{p}{(}\PY{n}{bagpack}\PY{o}{.}\PY{n}{volume} \PY{o}{+} \PY{l+m+mi}{1}\PY{p}{)}\PY{p}{:}
            \PY{k}{for} \PY{n}{item} \PY{o+ow}{in} \PY{n}{items}\PY{p}{:}
                \PY{k}{if} \PY{n}{w} \PY{o}{\PYZgt{}}\PY{o}{=} \PY{n}{item}\PY{o}{.}\PY{n}{weight} \PY{o+ow}{and} \PY{n}{v} \PY{o}{\PYZgt{}}\PY{o}{=} \PY{n}{item}\PY{o}{.}\PY{n}{volume}\PY{p}{:}
                    \PY{n}{table}\PY{p}{[}\PY{n}{w}\PY{p}{]}\PY{p}{[}\PY{n}{v}\PY{p}{]} \PY{o}{=} \PY{n+nb}{max}\PY{p}{(}\PY{n}{table}\PY{p}{[}\PY{n}{w}\PY{p}{]}\PY{p}{[}\PY{n}{v}\PY{p}{]}\PY{p}{,}
                                      \PY{n}{table}\PY{p}{[}\PY{n}{w} \PY{o}{\PYZhy{}} \PY{n}{item}\PY{o}{.}\PY{n}{weight}\PY{p}{]}\PY{p}{[}\PY{n}{v} \PY{o}{\PYZhy{}} \PY{n}{item}\PY{o}{.}\PY{n}{volume}\PY{p}{]} \PY{o}{+} \PY{n}{item}\PY{o}{.}\PY{n}{value}\PY{p}{)}

    \PY{n}{result} \PY{o}{=} \PY{p}{[}\PY{l+m+mi}{0}\PY{p}{]} \PY{o}{*} \PY{n+nb}{len}\PY{p}{(}\PY{n}{items}\PY{p}{)}
    \PY{n}{w} \PY{o}{=} \PY{n}{bagpack}\PY{o}{.}\PY{n}{weight}
    \PY{n}{v} \PY{o}{=} \PY{n}{bagpack}\PY{o}{.}\PY{n}{volume}
    \PY{k}{while} \PY{n}{table}\PY{p}{[}\PY{n}{w}\PY{p}{]}\PY{p}{[}\PY{n}{v}\PY{p}{]}\PY{p}{:}
        \PY{n}{aux} \PY{o}{=} \PY{p}{[}\PY{n}{table}\PY{p}{[}\PY{n}{w}\PY{o}{\PYZhy{}}\PY{n}{item}\PY{o}{.}\PY{n}{weight}\PY{p}{]}\PY{p}{[}\PY{n}{v}\PY{o}{\PYZhy{}}\PY{n}{item}\PY{o}{.}\PY{n}{volume}\PY{p}{]} \PY{o}{+} \PY{n}{item}\PY{o}{.}\PY{n}{value} \PY{k}{for} \PY{n}{item} \PY{o+ow}{in} \PY{n}{items}\PY{p}{]}
        \PY{n}{i} \PY{o}{=} \PY{n}{aux}\PY{o}{.}\PY{n}{index}\PY{p}{(}\PY{n}{table}\PY{p}{[}\PY{n}{w}\PY{p}{]}\PY{p}{[}\PY{n}{v}\PY{p}{]}\PY{p}{)}

        \PY{n}{result}\PY{p}{[}\PY{n}{i}\PY{p}{]} \PY{o}{+}\PY{o}{=} \PY{l+m+mi}{1}
        \PY{n}{w} \PY{o}{\PYZhy{}}\PY{o}{=} \PY{n}{items}\PY{p}{[}\PY{n}{i}\PY{p}{]}\PY{o}{.}\PY{n}{weight}
        \PY{n}{v} \PY{o}{\PYZhy{}}\PY{o}{=} \PY{n}{items}\PY{p}{[}\PY{n}{i}\PY{p}{]}\PY{o}{.}\PY{n}{volume}

    \PY{k}{return} \PY{n}{result}
\end{Verbatim}
\end{tcolorbox}

    \begin{tcolorbox}[breakable, size=fbox, boxrule=1pt, pad at break*=1mm,colback=cellbackground, colframe=cellborder]
\prompt{In}{incolor}{10}{\boxspacing}
\begin{Verbatim}[commandchars=\\\{\}]
\PY{k+kn}{import} \PY{n+nn}{time}

\PY{n}{start} \PY{o}{=} \PY{n}{time}\PY{o}{.}\PY{n}{time}\PY{p}{(}\PY{p}{)}

\PY{n}{max\PYZus{}items} \PY{o}{=} \PY{n}{knapsack}\PY{p}{(}\PY{n}{items}\PY{p}{,} \PY{n}{bagpack}\PY{p}{)}
\PY{n}{maxvalue}\PY{p}{,} \PY{n}{max\PYZus{}weight}\PY{p}{,} \PY{n}{max\PYZus{}volume} \PY{o}{=} \PY{n}{tot\PYZus{}value}\PY{p}{(}\PY{n}{max\PYZus{}items}\PY{p}{,} \PY{n}{items}\PY{p}{,} \PY{n}{bagpack}\PY{p}{)}
\PY{n}{max\PYZus{}weight} \PY{o}{=} \PY{o}{\PYZhy{}}\PY{n}{max\PYZus{}weight}
\PY{n}{max\PYZus{}volume} \PY{o}{=} \PY{o}{\PYZhy{}}\PY{n}{max\PYZus{}volume}

\PY{n+nb}{print}\PY{p}{(}\PY{l+s+s2}{\PYZdq{}}\PY{l+s+s2}{The maximum value achievable (by exhaustive search) is }\PY{l+s+si}{\PYZpc{}g}\PY{l+s+s2}{.}\PY{l+s+s2}{\PYZdq{}} \PY{o}{\PYZpc{}} \PY{n}{maxvalue}\PY{p}{)}
\PY{n}{item\PYZus{}names} \PY{o}{=} \PY{l+s+s2}{\PYZdq{}}\PY{l+s+s2}{, }\PY{l+s+s2}{\PYZdq{}}\PY{o}{.}\PY{n}{join}\PY{p}{(}\PY{n}{item}\PY{o}{.}\PY{n}{name} \PY{k}{for} \PY{n}{item} \PY{o+ow}{in} \PY{n}{items}\PY{p}{)}
\PY{n+nb}{print}\PY{p}{(}\PY{l+s+s2}{\PYZdq{}}\PY{l+s+s2}{  The number of }\PY{l+s+si}{\PYZpc{}s}\PY{l+s+s2}{ items to achieve this is: }\PY{l+s+si}{\PYZpc{}s}\PY{l+s+s2}{, respectively.}\PY{l+s+s2}{\PYZdq{}} \PY{o}{\PYZpc{}} \PY{p}{(}\PY{n}{item\PYZus{}names}\PY{p}{,} \PY{n}{max\PYZus{}items}\PY{p}{)}\PY{p}{)}
\PY{n+nb}{print}\PY{p}{(}\PY{l+s+s2}{\PYZdq{}}\PY{l+s+s2}{  The weight to carry is }\PY{l+s+si}{\PYZpc{}.3g}\PY{l+s+s2}{, and the volume used is }\PY{l+s+si}{\PYZpc{}.3g}\PY{l+s+s2}{.}\PY{l+s+s2}{\PYZdq{}} \PY{o}{\PYZpc{}} \PY{p}{(}\PY{n}{max\PYZus{}weight}\PY{p}{,} \PY{n}{max\PYZus{}volume}\PY{p}{)}\PY{p}{)}

\PY{n}{end} \PY{o}{=} \PY{n}{time}\PY{o}{.}\PY{n}{time}\PY{p}{(}\PY{p}{)}
\PY{n+nb}{print}\PY{p}{(}\PY{n}{f}\PY{l+s+s2}{\PYZdq{}}\PY{l+s+se}{\PYZbs{}n}\PY{l+s+s2}{The total time taken is }\PY{l+s+s2}{\PYZob{}}\PY{l+s+s2}{end\PYZhy{}start\PYZcb{}.}\PY{l+s+s2}{\PYZdq{}}\PY{p}{)}
\end{Verbatim}
\end{tcolorbox}

    \begin{Verbatim}[commandchars=\\\{\}]
The maximum value achievable (by exhaustive search) is 54500.
  The number of laptop, printer, headphone items to achieve this is: [9, 0, 11],
respectively.
  The weight to carry is 247, and the volume used is 247.

The total time taken is 0.20116853713989258.
    \end{Verbatim}

    ~\\
    \noindent\rule{16.5cm}{0.4pt}


\end{document}
